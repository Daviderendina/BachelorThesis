\section{Riconoscimento degli architectural smell su progetti reali}
    Questo capitolo presenta nel dettaglio tutte le attività svolte riguardanti l'esecuzione e la validazione degli algoritmi sviluppati su diversi progetti open-source. 
    Dopo una descrizione dei progetti, saranno analizzate nel dettaglio l'esecuzione degli algoritmi, le diverse attività di validazione svolte e i risultati ottenuti.  
    %Saranno inoltre approfondite anche tutte le operazioni per la validazione dei risultati, con descrizione delle differenti attività svolte e considerazioni approfondite sugli smell e sulle cause che hanno portato alla detection di falsi positivi. 
    Sarà inoltre effettuato un confronto sulle differenze della \textit{detection} di Arcan rispetto a quella del \textit{tool} Designite \cite{Designite}.
    Un'ultima sezione esporrà poi i differenti problemi affrontati durante lo svolgimento di queste attività.
    Prima della presentazione delle tematiche descritte in precedenza, sarà effettuata una breve digressione riguardo le attività di \textit{testing} svolte per la verifica degli algoritmi durante il loro sviluppo %e la correttezza dei risultati prodotti con input ben definiti. 

    \paragraph{Approccio Test Driven Development}
        Durante lo sviluppo degli algoritmi sono stati effettuati i loro test seguendo l'approccio \textit{Test Driven Development (TDD)} \cite{beck2003testdrivendev}. Questa strategia prevede la scrittura dei test anticipata rispetto all'implementazione degli algoritmi, in modo che lo sviluppo dell'applicativo sia orientato alla soddisfazione degli stessi. Al fine di applicare il \textit{TDD} è stato utilizzato il framework JUnit \cite{JunitWebsite}.
            % Inizialmente la prima fase di testing della capacità e correttezza degli algoritmi è stata durante lo sviluppo degli stessi, mediante l'utilizzo del framework JUnit \cite{JunitWebsite} per la creazione di test unitari. E' stato scelto infatti un approccio di Test Driven Development (TDD) \cite{beck2003testdrivendev}, che prevede la scrittura dei test prima dell'implementazione degli algoritmi in modo che lo sviluppo dell'applicativo abbia come obiettivo il passaggio di tutti i test unitari. 
        
        L'implementazione dei test unitari è stata fondamentale per la definizione degli obiettivi da raggiungere da parte degli algoritmi di \textit{detection}. Attraverso queste verifiche è stato possibile effettuare due tipologie di analisi differenti:
        \begin{itemize}
            \item \textit{Analisi di strutture sintetiche}, cioè grafi generati manualmente all'interno del codice dei test, utilizzati al fine di valutare il comportamento dell'algoritmo considerate diverse situazioni in termini di nodi e relazioni tra essi. In particolare attraverso le strutture sintetiche sono stati analizzati tutti i casi limite riguardanti le strutture del grafo, come ad esempio il numero di nodi coinvolti oppure le loro tipologie. L'obiettivo della scrittura di questi test è stato quindi il test su piccola scala, testando la capacità dell'algoritmo a riconoscere gli \textit{smell} in differenti scenari con un numero basso di nodi e archi e strutture ben definite.
            
            \item \textit{Analisi di piccoli progetti open source}, sono stati direttamente analizzati anche due progetti Java \textit{open source} di piccole dimensioni, Junit 4.13 \cite{JunitWebsite} e Jsoniter \cite{JsoniterWebsite}. Lo scopo dell'analisi di questi due progetti è stata la verifica, direttamente dal grafo generato da Arcan tramite la piattaforma Neo4J \cite{Neo4JWebsite}, che il numero degli \textit{smell} trovati attraverso \textit{query} Cypher \cite{Cypher} effettuate sul grafo fosse equivalente a quello trovato dagli algoritmi di \textit{detection}. Questi test quindi avevano come obiettivo la verifica che le strutture sintetiche riconosciute dall'algoritmo fossero individuate anche all'interno di progetti complessi e organizzati.
        \end{itemize}


    \subsection{Descrizione progetti utilizzati}
        Per le attività di individuazione degli \textit{smell} e validazione del lavoro svolto sono stati selezionati dieci progetti \textit{open-source} differenti, sviluppati da \textit{Apache Software Foundation} \cite{apache} e disponibili sulla piattaforma \textit{GitHub} \cite{github}. Una descrizione sommaria dei progetti è fornita dalla tabella 1, presentando informazioni riguardanti numero della release analizzata, dominio applicativo e collegamento al progetto disponibile su \textit{Github}.

A causa di alcuni problemi derivati dalle librerie utilizzate per il \textit{parsing} non è stato possibile analizzare tutti e dieci i progetti completi in tutti i loro componenti, perciò per alcuni di essi è stata necessaria l'analisi solamente del modulo principale del sistema. Nello specifico, per i progetti \textit{Druid}, \textit{Flink} e \textit{Geode} è stato utilizzato il modulo \textit{core} mentre per \textit{Beam} il modulo \textit{beam-runners core-java}.
Per l'analisi inoltre sono state escluse tutte le classi di test, in quanto considerate non molto significative ai fini del lavoro di ricerca degli \textit{smell} e loro validazione.

Informazioni più dettagliate sui progetti riguardanti modulo analizzato, righe di codice (\textit{LOC}) e numero di unit e package analizzati sono verificabili nella tabella 2. \\

% TABELLA DESCRIZIONE
\begin{table}[h]
    \centering
    \begin{tabular}{|c|c|c|c|c|}
        \hline
        \textbf{Progetto} & \textbf{Link} & \textbf{Release} & \textbf{Dominio applicativo}  \\
        \hline
        Accumulo & \cite{accumulogithub} & 2.0.0 & Distributed key/value data store  \\
        %\hline
        Beam & \cite{beamgithub} & 2.20.0 & Unified programming model  \\
        %\hline
        Bookkeeper & \cite{bookkeepergithub} & 4.10.0 & Storage service  \\
        %\hline
        Cassandra & \cite{cassandragithub} & 3.11.6 & Distributed NoSQL DBMS  \\
        %\hline
        Druid & \cite{druidgithub} & 0.18.0 & Distributed data store  \\
        %\hline
        Flink & \cite{flinkgithub} & 1.10.0 & Distributed processing engine  \\
        %\hline
        Geode & \cite{geodegithub} & 1.12.0 & In-memory data management system  \\
        %\hline
        Kafka & \cite{kafkagithub} & 2.5.0 & Distributed platform  \\
        %\hline
        Skywalking & \cite{skywalkinggithub} & 7.0.0 & Application performance monitor system  \\
        %\hline
        Zookeeper & \cite{zookeepergithub} & 3.6.0 & Services for distributed system  \\
        \hline
    \end{tabular}
    \caption{Informazioni progetti analizzati}
    \label{tab:caption}
\end{table}
\defaultvspace
\begin{table}[h]
    \centering
    \begin{tabular}{|c|c|c|c|c|c|}
        \hline
        \textbf{Progetto} & \textbf{Modulo} & \textbf{N\textsuperscript{o} classi} & \textbf{N\textsuperscript{o} package} & \textbf{N\textsuperscript{o} LOC}  \\
        \hline
        Accumulo & - & 3 789 & 212 & 43 682 818 \\
        Beam & core-java & 201 & 8 & 52 414 \\
        Bookkeeper & - & 2 036 & 252 & 1 495 659 \\
        Cassandra & - & 3 910 & 116 & 33 968 645 \\
        Druid & druid-core & 736 & 63 & 369 487 \\
        Flink & flink-core & 915 & 49 & 148 600 \\
        Geode & geode-core & 4 293 & 190 & 2 236 679\\
        Kafka & - & 2 389 & 138 & 575 967 \\
        Skywalking & - & 979 & 526 & 141 674 \\
        Zookeeper & - & 755 & 52 & 269 378 \\
        \hline
    \end{tabular}
    \caption{Dettaglio progetti analizzati}
    \label{tab:caption}
\end{table}


    \subsection{Risultati della detection}
        L'analisi dei progetti è stata effettuata attraverso l'esecuzione di Arcan, in grado di generare in \textit{output} diversi file in formato \textit{csv} per ogni sistema analizzato. I file generati per un singolo progetto forniscono informazioni riguardanti:
\begin{itemize}
    \item I valori delle metriche (come \textit{instability, LOC}) calcolate su tutti gli elementi presenti nell'applicativo.
    
    \item Le unit colpite dallo \textit{smell} (con relazione \textit{affects} verso il nodo che lo rappresenta) e la tipologia riscontrata.
    
    \item Tutte le unit collegate ad un nodo di tipo \textit{smell}, con indicazione sulla tipologia di relazione tra i due elementi.
\end{itemize}
L'analisi di questi tre \textit{file} ha permesso lo studio e la validazione dei risultati ottenuti dagli algoritmi.

La \textit{detection} degli \textit{smell} sui progetti \textit{Apache} ha evidenziato la presenza di numerose istanze. In particolare sono stati rilevate 708 \textit{Subclasses Do Not Redefine Methods}, 2033 \textit{Unutilizied Abstraction} e 2491 \textit{Unnecessary Abstraction}. Il dettaglio riguardante il numero di \textit{smell} trovati nei vari progetti è presentato nella tabella 3. 

Come si evince dai risultati, \textit{Subclasses Do Not Redefine Methods} è lo \textit{smell} che presenta il numero minore di istanze; la causa principale di ciò può essere identificata nel numero differente di elementi analizzati per la ricerca. Infatti \textit{UUA} e \textit{UNA} effettuano la loro \textit{detection} controllando ogni \textit{abstraction} all'interno del progetto, mentre questo \textit{smell} considera solamente le gerarchie presenti (e le relative \textit{unit} coinvolte), che sono sicuramente in numero minore rispetto alle classi e interfacce del sistema. 
\defaultvspace
\begin{table}[h]
    \centering
    \begin{tabular}{|c|c|c|c|}
        \hline
        \textbf{Progetto} & \textbf{Subclasses Do Not} & \textbf{Unutilizied} & \textbf{Unnecessary}  \\
         & \textbf{Redefine Methods} & \textbf{Abstraction} & \textbf{Abstraction}\\
        \hline
        Accumulo & 98 & 177 & 1113 \\
        Beam & 5 & 32 & 13 \\
        Bookkeeper & 52 & 189 & 96  \\
        Cassandra & 96 & 137 & 639 \\
        Druid & 6 & 128 & 52 \\
        Flink & 66 & 185 & 47 \\
        Geode & 143 & 304 & 166  \\
        Kafka & 138 & 134 & 119 \\
        Skywalking & 65 & 683 & 208\\
        Zookeeper & 39 & 64 & 38 \\
        \hline
        Totale & 708 & 2033 & 2491 \\
        \hline
    \end{tabular}
    \caption{Numero di istanze individuate nei progetti}
    \label{tab:caption}
\end{table} 
\defaultvspace \\
%
Per quanto riguarda \textit{Unutilizied Abstraction} invece si può affermare che la presenza di un alto numero di istanze può essere causato da due fattori principali. La semplicità con la quale questo \textit{smell} può essere introdotto nel design è sicuramente uno di questi, poiché le cause della sua manifestazione sono molto comuni nello sviluppo software.
Inoltre la presenza di \textit{UUA} può verificarsi anche in seguito alla manifestazione "effetto collaterale" della presenza di \textit{Unnecessary Abstraction}. Un esempio può essere la creazione di una classe \textit{constant placeholder}, quando la \textit{abstraction} è rappresentata da un'interfaccia che non viene implementata ma solamente utilizzata via \textit{dot notation}. In questo caso il nodo non avrà alcun arco in ingresso ad eccezione di \textit{dependsOn}, facendo risultare l'interfaccia sia come \textit{unnecessary} che \textit{unutilizied}.

In merito a \textit{Unnecessary Abstraction} ritengo che 
%la difficoltà ad inquadrare una situazione ben precisa dello
la definizione non molto dettagliata dello \textit{smell} e soprattutto la necessità di identificare il caso delle classi procedurali abbiano portato ad un elevato numero di istanze individuate. La situazione descritta da \textit{UNA} è un po' ambigua, siccome non espone come gli altri una scenario ben definito ma si riferisce genericamente a classi "non necessarie per il design". Attraverso le cause che lo introducono \cite{SURYANARAYANA201521} è stata possibile la definizione di tre sue diverse tipologie, ma è comunque presente il caso delle \textit{procedural class}
che risulta molto difficoltoso da identificare, poiché le classi procedurali derivano principalmente dalle intenzioni dello sviluppatore piuttosto che dalla sua struttura e dai metodi e attributi definiti.
La loro ricerca ha influenzato in maniera significativa l'alto numero di \textit{Unnecessary Abstraction} presenti, come evidenziato dai dati riportati nella tabella 4, dove si evince che il 92\% dei casi riportati sono dovuti proprio a questa categoria di classi (2275 istanze su 2491 totali), a differenza di \textit{constant placeholder} e \textit{over engineered} che hanno un incidenza sul totale rispettivamente del 7\% e 1\% (con 178 e 17 casi individuati). 
L'alto di numero di classi procedurali è a sua volta influenzato fortemente dal progetto Accumulo, dove sono presenti in totale 1113 classi identificate come \textit{smell}, che rappresentano il 44\% dei casi totali riscontrati su tutti i progetti. Anche qui la casistica più frequente è quella delle \textit{procedural class}, con ben 996 istanze trovate. 

\begin{table}[h]
    \centering
    \begin{tabular}{|c|c|c|}
        \hline
        \textbf{Tipologia} & \textbf{Istanze smell} & \textbf{Incidenza totale}\\
        \hline
        Constant placeholder & 178 & 7\% \\
        Over engineered & 17 & 1\% \\
        Procedural class & 2275 & 92\% \\
        \hline
    \end{tabular}
    \caption{Dettaglio istanze smell Unnecessary Abstraction}
    \label{tab:caption}
\end{table}


  
    
    


        
    \subsection{Validazione dei risultati}
            %Introduzione e spiegazione concetti fondamentali
    Questa sezione descrive in dettaglio la fase di validazione degli algoritmi, con descrizione delle attività svolte e discussione dei risultati ottenuti.
    %Questa sezione descrive nel dettaglio le attività svolte e i risultati ottenuti dalla fase di \textit{validation} degli algoritmi.
    Prima di approfondire questi aspetti è però necessaria l'introduzione di alcuni concetti fondamentali, al fine di favorire la comprensione delle attività descritte.
    
    %Veri e falsi positivi
    I primi concetti introdotti sono \textit{true positive} (vero positivo, VP) e \textit{false positive} (falso positivo, FP). Con il termine \textit{true positive} si indicano le istanze di smell trovate dagli algoritmi che risultano come problemi reali mentre \textit{false positive} indica il contrario, ovvero tutte le istanze individuate che però non risultano tali. 
    
    %Metrica Precision
    Il numero di veri e falsi positivi trovati da un algoritmo è fondamentale per il calcolo della metrica \textit{Precision} \cite{wikiPrecisionRecall}. Questa metrica è indice della precisione dell'algoritmo sviluppato in termini di numero di veri positivi in rapporto al totale di casi analizzati e può essere descritta dalla formula seguente:
    $$Precision = { \mbox{\textit{false positives}} \over \mbox{\textit{true positives}} + \mbox{\textit{false positives}}}$$
    
%Come è stata fatta la validazione? 
    Per lo svolgimento delle attività di validazione sono state selezionate un numero fisso di 50 classi per ogni \textit{smell} in ogni progetto, segnalate dall'algoritmo di \textit{detection}. Nei casi in cui il numero di istanze fosse minore, sono state prese in considerazione tutte quelle presenti anche in numero inferiore.
    
    Ogni elemento coinvolto è stato analizzato manualmente attraverso l'osservazione del codice sorgente, allo scopo di verificare la presenza effettiva dello \textit{smell} e analizzare le cause che hanno portato l'elemento a essere considerato come tale. Per la validazione di \textit{Unnecessary Abstraction} sono state prese, ove possibile, un numero uguale di classi per ogni tipologia presente. Le successive analisi presenteranno i risultati ottenuti dalla \textit{detection} di \textit{UNA} considerando sia i valori generici ottenuti per lo \textit{smell} sia i tre casi distinti.
    
    Lo studio dei falsi positivi riscontrati ha fatto emergere la necessità di effettuare modifiche agli algoritmi e agli algoritmi di \textit{parsing}, al fine di rimuovere alcune tipologie ricorrenti di falsi positivi e migliorare la \textit{precision} della \textit{detection}. 
 
  %  Validazione incrociata? 
    Al fine di garantire una maggior precisione nella validazione degli algoritmi, è stata effettuata un'attività di validazione incrociata con un'altra laureanda \cite{rotaPhdTdthesis}. Questa attività ha portato a un'analisi e validazione doppia dei falsi positivi, con successive valutazioni e verifica dei casi dove il risultato delle due non coincideva. %Scrivo i benefici che ha portato?
    
   La tabella 5 presenta in maniera sintetica i risultati ottenuti dalla validazione delle \textit{detection} degli algoritmi \textit{SR}, \textit{UNA} e \textit{UUA}, e fornisce informazioni sul numero totale di \textit{smell} analizzati e il numero di falsi positivi riscontrati. Un'ulteriore colonna indica il valore della metrica \textit{Precision} per l'algoritmo considerato. 
   \defaultvspace
    \begin{table}[h]
            \centering
            \begin{tabular}{|c|c|c|c|c|}
                \hline
                \textbf{Smell} & \textbf{Analizzati} & \textbf{ VP } & \textbf{ FP } & \textbf{Precision} \\
                \hline
                Subclasses Do Not Redefine M. & 400 & 359 & 41 & 89.75\% \\ 
                Unutilizied Abstraction & 460 & 439 & 21 & 95.43\% \\
                Unnecessary Abstraction & 448 & 330 & 118 & 73.66\% \\
                \hline
        \end{tabular}
        \caption{Panoramica validazione}
        \label{tab:caption}
    \end{table}
    \defaultvspace \\
    Le tre sottosezioni successive analizzeranno nel dettaglio la validazione di ogni singolo \textit{smell}. Verrà posta maggior enfasi sulle tecniche utilizzate, sui \textit{tool} impiegati come supporto e saranno approfondite le diverse cause della presenza di falsi positivi.
    
    
    %Subclasses does not redefine
    \subsubsection{Subclasses Does Not Redefine Methods}
        Per effettuare la validazione di \textit{Subclasses Do Not Redefine Methods} sono state analizzate le superclassi di ogni unit segnalata dall'algoritmo come istanza, al fine di verificare che le due classi condividessero almeno un metodo. Fondamentale è stato il supporto fornito dall'ambiente di sviluppo IntelliJ IDEA \cite{intelliJ}, in grado di fornire informazioni riguardanti l'\textit{override} delle funzioni all'interno delle diverse classi o interfacce. 
        
        La \textit{precision} calcolata su questo algoritmo è del 89.75\%, con 359 veri positivi su un totale di 400 unit analizzate.
        % Modifiche alla detection dopo prima validazione
        %Una prima fase di validazione aveva visto questo algoritmo avere una precision del 73\%, con 418 istanze validate di cui 307 vere positive. 
        Una prima validazione di \textit{SR} aveva mostrato un valore di \textit{precision} del 73\%, con 418 istanze validate di cui 307 veri positivi.
        Al termine di questa fase è stata notata la presenza di numerosi falsi positivi, che ha portato allo studio dei casi riscontrati al fine di effettuare modifiche per favorire la diminuzione del loro numero e migliorare la precisione dell'algoritmo.
        %e si è optato per la rimozione degli stessi al fine di migliorare la \textit{precision} dello stesso. 
        A tal proposito, è stata introdotta un'unica modifica alle strategie di detection: nella ricerca dei metodi ereditati dal supertipo, sono stati considerati anche tutti i metodi delle interfacce implementate da classi astratte.
        Questa modifica ha permesso l'aumento del valore di \textit{precision} di circa 17 punti percentuali, passando da 307 VP su 418 istanza analizzate a 359 VP su un totale di 400. La correzione effettuata ha portato inoltre alla eliminazione di 166 istanze totali, di cui 78 falsi positivi e 5 veri positivi (dei restanti non si hanno informazioni perché non selezionati per la validazione).
        %. Tra gli smell validati si è assistito quindi a una rimozione del 94\% di falsi positivi e questa percentuale così alta fa pensare che anche tra gli smell non validati fossero presenti molti casi di falsa positività.
        
        \paragraph{Analisi dei falsi positivi}
            Nonostante le modifiche effettuate, è stata riscontrata la presenza di ulteriori situazioni di istanze risultate come false positive, che per diversi motivi non è stato possibile eliminare.
            %Sono presenti alcune situazioni nelle quali l'algoritmo segnala la presenza di falsi positivi; in generale, si è verificato nel xpercento dei casi per questo smell. 
            Tra le differenti situazioni, verrà approfondita solamente quella relativa alla estensione delle classi esterne al sistema, poiché è stata riscontrata con una certa rilevanza durante la validazione. 
            Siccome i componenti di Arcan, in particolare il \textit{System Reconstructor}, costruiscono il grafo solamente a partire dal codice sorgente ricevuto come input, diverse dipendenze esterne non vengono considerate nella generazione del grafo.
            %Quando una classe di sistema è inclusa in una gerarchia, può verificarsi che sia presente una gerarchia di classi derivate tutte da essa. 
            Può verificarsi il caso di presenza nel sistema di una gerarchia, che presenta come primo elemento una classe di sistema.
            Se una classe di questa gerarchia, non figlia direttamente della classe di sistema, ridefinisce solamente i suoi metodi, viene considerata erroneamente come \textit{smell} poiché quest'ultima non è presente nel grafo in quanto dipendenza esterna. Se invece si considera una classe figlia direttamente di una classe esterna, la relazione tra le due classi non viene considerata nella ricerca dello \textit{smell} poiché non viene rappresentata nel grafo.
            %Quando una classe di sistema è inclusa in una gerarchia si possono verificare due casi differenti. Se una classe ne estende un'altra che a sua volta è figlia di una classe di sistema, e l'ultima classe della gerarchia ridefinisce solamente metodi della classe di sistema, viene considerato erroneamente come smell, poiché non essendo presente il nodo della classe di sistema nel grafo non vengono considerati i suoi metodi. Se una classe invece estende direttamente una di sistema, questa gerarchia non viene considerata come smell a prescindere dal fatto che ridefinisca o meno i suoi metodi, poichè la gerarchia non è presente nel grafo.
            Facendo un esempio del primo caso, se una classe B estende una classe A che a sua volta è sottotipo di Beans \cite{javadocWebsite} e B effettua \textit{override} del metodo \textit{instantiate} di Beans, la classe B risulterebbe erroneamente come contenente lo \textit{smell} \textit{SR}.
            
           % \item \textit{Metodi astratti} Il parser utilizzato \cite{pawlak:hal-01169705} non riconosce i metodi astratti nelle classi e perciò non vengono aggiunte alle varie classi. Questo ovviamente comporta un problema quando una classe ne estende un'altra e ridefinisce un suo metodo astratto poichè, non essendoci il metodo astratto nel grafo, non viene riconosciuto come override e quindi risulta all'algoritmo come un AS (?). %Ridefinisce i metodi dell'interfaccia implementata dal padre, che però è abstract e quindi può anche non ridefinirli (considerato come VP ma mooolto indeciso)
            
            \begin{table}[h]
                \centering
                \begin{tabular}{|c|c|c|c|c|}
                    \hline
                    \textbf{Progetto} & \textbf{Analizzati} & \textbf{Veri positivi} & \textbf{Falsi positivi} \\
                    \hline
                    Accumulo & 50 & 50 & 0 \\
                    Beam & 5 & 4 & 1 \\
                    Bookkeeper & 50 & 45 & 5  \\
                    Cassandra & 50 & 43 & 7 \\
                    Druid & 6 & 6 & 0 \\
                    Flink & 50 & 49 & 1 \\
                    Geode & 50 & 48 & 2 \\
                    Kafka & 50 & 50 & 0 \\
                    Skywalking & 50 & 49 & 1 \\
                    Zookeeper & 39 & 15 & 24 \\
                    \hline
                    Totale & 400 & 359 & 41 \\
                    \hline
                \end{tabular}
                \caption{Risultati validazione Subclasses Do Not Redefine Methods}
                \label{tab:caption}
            \end{table}
    
    %Unutilizied Abstraction
    %UUA
    \subsubsection{Unutilizied Astraction}
        La validazione di \textit{Unutilizied Abstraction} è stata effettuata verificando gli utilizzi di ogni classe e interfaccia segnalata come istanza dello \textit{smell} all'interno del progetto. I casi risultati come falsi positivi hanno comportato un'ulteriore analisi della struttura del grafo, al fine di comprendere le motivazioni che hanno portato alla loro \textit{detection}.
        % La valiudazione di questo smell è stata effettuata controllando gli utilizzi di ogni classe o interfaccia coninvolta nel progetto; i casi che sono risultati come FP hanno visto anche un'approfondimento della struttura del grafo, al fine di comprendere i motivi che hanno portato a questi FP.
        
        Uno strumento fondamentale durante la validazione è stato l'ambiente di sviluppo IntelliJ IDEA \cite{intelliJ}. Grazie allo strumento messo a disposizione da questo \textit{IDE} per la ricerca delle \textit{references} di una classe o interfaccia, è stato possibile ottenere informazioni riguardanti gli utilizzi dell'elemento analizzato all'interno del progetto.\\
            %, è stato possibile ricercare le \textit{references} nel progetto di una determinata classe o interfaccia ottenendo anche indicazioni sull'utilizzo effettuato. con la conseguenza che non è stata necessaria la verifica di tutte le references nel codice. 
        \\
        La \textit{precision} riscontrata sull'algoritmo è del 95.43\%, con 418 veri positivi su 439 istanze analizzate. Grazie all'alta percentuale riscontrata, non è stato ritenuto necessario effettuare modifiche agli algoritmi successive alla prima validazione.
            %Le precisione dell'algoritmo è molto soddisfacente e, a differenza degli altri due smell, non sono state effettuate modifiche successive alla prima validazione in quanto i risultati sono stati giudicati soddisfacenti.
        
        \paragraph{Analisi di falsi positivi}
            La maggioranza delle istanze riscontrate è riferito a classi di esempio, che sono parte del progetto ma non vengono utilizzate da nessuno.
            % Il numero di falsi positivi riscontrati in questo smell è molto esiguo. La maggioranza di queste istanze è dato da classi relativi ad esempi, che ovviamente fanno parte del progetto ma non vengono utilizzate da nessuno. 
            Un ulteriore causa identificata che ha portato la presenza di falsi positivi è legata al \textit{parsing}, poiché alcune \textit{references} di classi trovate nel codice non hanno trovato riscontro nel grafo delle dipendenze. È stato notato in particolare che gli utilizzi all'interno di classi anonime non vengono rilevati dal \textit{tool}, in quanto nel grafo non sono presenti i nodi rappresentanti classi anonime e le relative dipendenze.
            
            % Altre cause che hanno portato alla presenza di queste istanze sono dovuti tutti al parsing degli elementi ed al grafo e alla mancanza nello stesso di dipendenze tra gli elementi. 
            %In particolare, si segnala che l'utilizzo da parte di classi anonime non viene rilevato dal tool, poichè nel grafo non sono presenti i nodi rappresentanti le classi anonime e quindi nemmeno le dipendenze tra quest'ultime e le classi utilizzate. 
            \defaultvspace
            \begin{table}[h]
            \centering
                \begin{tabular}{|c|c|c|c|c|}
                    \hline
                    \textbf{Progetto} & \textbf{Analizzati} & \textbf{Veri positivi} & \textbf{Falsi positivi} \\
                    \hline
                    Accumulo & 50 & 44 & 6 \\
                    Beam & 32 & 32 & 0 \\
                    Bookkeeper & 50 & 50 & 0  \\
                    Cassandra & 50 & 50 & 0 \\
                    Druid & 50 & 49 & 1 \\
                    Flink & 50 & 50 & 0 \\
                    Geode & 50 & 48 & 2 \\
                    Kafka & 50 & 41 & 9 \\
                    Skywalking & 50 & 47 & 3 \\
                    Zookeeper & 28 & 28 & 0 \\
                    \hline
                    Totale & 460 & 439 & 21 \\
                    \hline
                \end{tabular}
                \caption{Risultati validazione Unutilizied Abstraction}
                \label{tab:caption}
            \end{table}
        
    %Unnecessary Abstraction
    \subsubsection{Unnecessary Abstraction}
    Le attività e i risultati della validazione di \textit{Unnecessary Abstraction} verranno presentati sia considerando lo \textit{smell} dal punto di vista generico che analizzando i tre casi distinti dello stesso. I valori di \textit{precision} calcolati per le tre differenti categorie possono essere osservati nella tabella 8.
    %In questo senso, la tabella 8 mostra nel dettaglio i valori di \textit{precision} per le tre casistiche considerate. 
    
    La validazione di questo \textit{smell} ha presentato diversi problemi, a causa di numerosi casi particolari e difficoltà nell'ideazione di strategie e sviluppo di algoritmi, testimoniate dal valore di \textit{precision} di 73.66\%, il minore tra i tre calcolati. Come mostrato dalla tabella 8, la precisione dell'algoritmo è fortemente influenzata dal caso \textit{procedural class}, che presenta un alto numero di istanze analizzate e una precisione del 65.33\%. La precisione dei casi \textit{constant placeholder} e \textit{over engineered} invece assume i valori rispettivamente del 86.75\% e 87.50\%.   

    
    %Come ho fatto la validation?
    Le tre casistiche dello \textit{smell} hanno richiesto differenti attività per la validazione dei risultati.
    Per i casi \textit{constant placeholder} e \textit{over engineered}, sono state analizzate le classi per verificare che seguissero i vincoli descritti nel capitolo precedente. Per le \textit{procedural class} invece il controllo riguardava non solo il rispetto o meno dei vincoli, ma anche le motivazioni dello sviluppo della classe per comprendere se la stessa potesse essere considerata come \textit{smell} o meno. Per fare questo, sono stati analizzati gli attributi utilizzati dalle diverse funzioni: venivano giudicate \textit{procedural class} tutte quelle che svolgevano una banale trasformazione dei dati in ingresso e producevano in uscita un risultato, mentre classi che effettuavano trasformazioni al sistema oppure ad altri oggetti in modo permanente non venivano considerate tali. Un ulteriore verifica comprende l'eventuale struttura nella quale era inserita, i package e le interfacce implementate, i commenti alla classe e alle funzioni e i nomi delle funzioni stesse. 

    
    %Modifiche alla validation
    Anche questa \textit{detection}, come già effettuato per \textit{Subclasses Do Not Redefine Methods}, ha subito un adattamento al termine della prima fase di validazione. Prima delle modifiche agli algoritmi, la \textit{precision} calcolata su questo algoritmo era 59.66\%, con 278 veri positivi su 466 totali. Si è verificato quindi un incremento di circa 14 punti percentuale, portandolo all'attuale valore di 73.66\%. 
    Il numero di istanze segnalate da Arcan è inoltre calato di 722 unità, con la rimozione di 134 istanze selezionate precedentemente per la prima \textit{validation}, tra le quali si registrano 8 veri positivi e 126 falsi positivi (94\% dei casi validati e rimossi). 
    
    Nelle tre casistiche introdotte per lo \textit{smell}, sono state effettuate le modifiche elencate di seguito:
    \begin{itemize}
        \item Per la ricerca delle \textit{constant placeholder} non sono state più considerate le classi che possiedono un supertipo. Questa modifica ha comportato un incremento della \textit{precision} da 78.57\% a 86.75\%, con un passaggio da 136 VP su 168 totali a 144 VP su 166 attuale.
            %\textit{Constant placeholder} ha visto l'esclusione delle classi che presentano supertipi dalla detection, con un aumento della \textit{precision} da 78.57\% a 86.75\% e con un passaggio da 136 VP su 168 totali a 144VP su 166.
         
        \item L'introduzione di limiti più stringenti per la \textit{detection} di \textit{over engineered} ha fatto registrare un aumento della precision da 27.59\% a 87.50\%. La differenza molto grande in termini di punti percentuale è dovuta al numero esiguo di istanze trovate, con un numero di casi considerati prima e dopo le modifiche effettuate rispettivamente di 29 e 8. I cambiamenti introdotti riguardano il controllo del nome delle funzioni, che viene fatto per intero e non solamente controllando che il nome inizi con le stringhe \textit{"get"} oppure \textit{"set"}. Viene verificato inoltre che le classi sospette non ereditino nulla da eventuali supertipi. 
            % In una situazione over engineered anche se il supertipo fosse vuoto si considera (e non si eliminano come gli altri casi) poichè potrebbe essere tutto over engineered, anche la generalizzazione
            %\textit{Over engineered} invece ha visto aumentare la metrica \textit{precision} da 27.59\% a 87.50\%, grazie all'introduzione di limiti più stringenti per quanto riguarda il controllo dei nomi delle funzioni (si è passato da controllare solamente se il nome delle funzioni iniziasse con 'get' oppure 'set' al controllo che effettivamente la funzione presentasse un nome del tipo getNomeAttributo oppure setNomeAttributo) oppure al non considerare classi che ereditavano metodi o attributi dai loro supertipi. L'aumento di così tanti punti percentuale è dovuto soprattutto al fatto che le classi considerate sono molte poche, 29 nella prima validation e solamente 8 nella seconda.
        
        
        \item La rimozione delle classi che presentavano supertipi nella ricerca delle \textit{procedural class} ha comportato un aumento della metrica \textit{precision} da 51.30\% a 65.33\%, passando da 138 VP su 269 istanze a 179 VP su 274. 
            %Il valore di \textit{precision} molto più basso rispetto a tutti gli altri riscontrati è indice del fatto che, come già anticipato in precedenza, è stato in assoluto il caso più critico di cui fare la \textit{detection} e \textit{validation}.
    \end{itemize}
    %
    \defaultvpsace
    \begin{table}[h]
        \centering
        \begin{tabular}{|c|c|c|c|c|}
            \hline
            \textbf{Caso} & \textbf{Analizzati} & \textbf{VP} & \textbf{FP} & \textbf{Precision} \\
            \hline
            Constant Placeholder & 166 & 144 & 22 & 86.75\%\\
            Over-engineered & 8 & 7 & 1 & 87.50\%\\
            Procedural classes & 284 & 179 & 95 & 65.33\% \\
            \hline
        \end{tabular}
        \caption{Dettaglio casistica Unnecessary Abstraction}
        \label{tab:caption} 
    \end{table}
    
    \paragraph{Analisi di falsi positivi}
        Le cause della presenza di numerose istanze risultate come falsi positivi può essere ricercata in tre fattori fondamentali: 
        \begin{enumerate}
            %\item Tra i tre \textit{smell} introdotti, è quello con la definizione meno specifica, senza una definizione precisa di una particolare situazione e con diverse casistiche presenti. Di conseguenza, le regole di detection si sono rilevate molto più difficili da implementare.
            
            \item A differenza degli altri due \textit{smell}, \textit{UNA} presenta un gran numero di casi particolari e situazioni limite, emerse soprattutto durante le fasi di validazione. Di conseguenza le regole di detection sono risultate molto più difficili da implementare, soprattutto nell'ottica di inserire limiti stringenti per le varie casistiche.
            
            %\item Le regole di detection si sono rivelate molto più leggere rispetto agli altri due casi, con conseguenza un numero significativo di classi trovate che rientravano nelle caratteristiche ma non erano affette dallo \textit{smell}. 
            
            \item Le difficoltà riscontrate nella detection del caso \textit{procedural class} hanno un impatto considerevole sul valore di \textit{precision} calcolato per l'algoritmo di detection, in quanto è il caso più rappresentato dei tre con 62\% delle istanze validate di \textit{Unnecessary Abstraction} e presenta il valore di \textit{precision} minore (65.33\%).
        \end{enumerate}
        %
        Al fine di approfondire le cause della presenza di falsi positivi nel dettaglio, saranno ora analizzati i risultati della validazione delle diverse tipologie singolarmente.
            % 3. Verranno ora analizzati i risultati della validazione delle tipologie di \textit{Unnecessary Abstraction} individualmente, al fine di approfondire le cause nel dettaglio.
            % 1. Dopo aver analizzato i problemi principali che hanno portato a questa situazione, si approfondiscono i casi nel dettaglio. 
            % 2. Innanzitutto, penso sia fondamentale scindere le tre casistiche per discuterle separatamente, in quanto molto differenti tra loro.
            %\begin{itemize}
        
        La causa più comune della presenza di falsi positivi \textit{constant placeholder} è rappresentata da classi che presentano pochi attributi, generalmente da 1 a 3, non considerate come placeholder in quanto contenenti campi di significati e tipologie differenti o diverse istanze di classi anonime. Spesso le classi presentano solamente un solo attributo come unica istanza di un oggetto, una situazione simile al design pattern \textit{Singleton} \cite{gamma1994design}. 
            %un solo attributo costante, spesso rappresentante un unica istanza dell'oggetto della classe (simile al pattern \textit{Singleton}), o \textit{abstraction} che presentano pochi attributi (1-3) ma non sono state considerate \textit{placeholder} dal momento che contengono campi di tipi diversi e spesso istanze di classi anonime.
            %sicuramente il caso più critico è quello di classi che presentano solo un attributo, spesso rappresentante un istanza unica dell'oggetto della classe. Un altra situazione comune sono classi che definiscono pochi attributi, in genere da 1 a 3, ma non rappresentano un placeholder per costanti, poichè gli attributi presenti presentano tipi diversi e scopi differenti e non sono propriamente valori ma per di più oggetti di classi anonime.
            % Altre situazioni che hanno portato alla presenza di falsi positivi sono la presenza nella classe di pochi attributi (1-3) ma non rappresentano un vero e proprio placeholder per costanti, perchè presentano anche tipi diversi e scopi ben differenti. 
        Una possibile soluzione per la loro rimozione potrebbe essere l'introduzione di una soglia minima di attributi per essere considerata un \textit{placeholder}, che porterebbe però a una possibile limitazione dell'algoritmo siccome potrebbero rimanere escluse dall'algoritmo diverse classe considerate ad oggi come vere positive.% verrebbero tutte le istanze con pochi attributi considerate però vere positive.
            % problemi vecchi - con queste regole di detection, la ricerca delle costant placeholder porta anche diversi falsi positivi. Un caso molto frequente è quello di classi che hanno solamente un parametro costante che rappresenta un ID statico e un costruttore che richiama quello del suo supertipo; siccome del costruttore delle classi non viene fatto il parsing, questa risulta come una classe senza metodi e con solo attributi costanti. Una soluzione a questo problema potrebbe essere il parsing del costruttore, ma questo porterebbe sicuramente molti VP a non essere più considerati tale, poichè ho riscontrato la presenza di numerose classi costant-placeholder con il costruttore privato e vuoto. Un'altra soluzione potrebbe quindi consistere nell'inserimento di un controllo che la classe non sia un sottotipo.
            
        L'esiguo numero di istanze \textit{over engineered} dello \textit{smell} e di falsi positivi riscontrati (rispettivamente 8 e 1) non rendono possibile alcuna analisi su di essi. L'unico falso positivo è relativo a una classe che presenta tutte le caratteristiche che le fanno considerare \textit{over engineered} ma contiene un'altra classe al suo interno. Questo causa anche che il nodo della dipendenza in ingresso alla classe sia in realtà dato dalla sua classe interna (nel grafo ogni classe interna ha un arco \textit{dependsOn} verso la unit che la contiene), rendendola così un falso positivo. 
            % problemi vecchi - la detection di questa parte degli smell è inadeguata, infatti si è pensato di rimuoverla. Anche qui si potrebbero controllare eventuali classi padre (che eliminerebbero lo smell) e inoltre bisognerebbe mettere un controllo più stringente sui metodi, ovvero dovrebbero avere esattamente il nome getAttributo e setAttributo; in pratica, si andrebbe solamente a cercare quel caso nello specifico.
            
        Infine le \textit{procedural class} presentano caratteristiche molto comuni anche per classi non procedurali, causando così la \textit{detection} di numerose istanze considerate poi come falsi positivi. Anche se le classi presentano tutte le caratteristiche necessarie per essere considerate procedurali, può succedere che molto spesso non siano comunque ritenute tali poiché sono presenti diversi fattori esterni alla sola struttura della classe difficoltosi da analizzare attraverso un algoritmo di analisi su un grafo.
        Durante la valutazione delle istanze di questo algoritmo vengono infatti presi in considerazione aspetti quali nomi assegnati agli elementi, codice ed eventuali commenti presenti, che influiscono in maniera significativa sulla valutazione della classe come procedurale o meno. %, difficili da analizzare attraverso un algoritmo.
        %L'analisi delle intenzioni del programmatore durante la generazione della classe infatti, senza poter valutare elementi come codice, nomi di classi e funzioni ed eventuali commenti, può risultare molto complicata attraverso un algoritmo. 
        Inoltre in un progetto la presenza di alcune classi procedurali è comune, come ad esempio le classi che rappresentano \textit{Thread} oppure quelle che che implementano il metodo \textit{main}, ma queste vengono individuate erroneamente da Arcan come \textit{smell}, generando casi di falsa positività.
            % Ad esempio è difficoltoso capire le intenzioni del programmatore che l'hanno portato a generare quella classe in quel particolare modo, se non analizzandone il codice, i nomi dati alla classe e alle diverse funzioni ed eventuali commenti lasciate dagli sviluppatori.\\
            % Questo caso presenta il più basso valore di precision tra tutti; questo è causa del fatto che è in assoluto il più difficile di cui fare la detection (come gia spiegato prima)(anche se 'rientra' nei vincoli, ci sono fattori esterni alla struttura del programma e soggettivi che è difficoltoso esprimere con un algoritmo - come faccio a capire se la classe è intesa come porcedurale o no solamente analizzando un grafo? è molto complesso.) 
            % problemi vecchi - un problema principale per la detection di queste classi è quello che sono molto comuni. Un esempio di FP ricorrente sono le Factory, spesso riconosciute come Procedural class ma in realtà hanno uno scopo ben preciso; si potrebbe filtrare il nome delle classi, sulla falsariga di quanto fatto per le Exception in SR (?), per risolvere questo problema.Anche qui inoltre si potrebbe abbassare il numero dei FP controllando le classi che utilizzino la loro classe padre ed escluderle così dalla lista degli smell, caso che si è verificato molto spesso.
         \defaultvpsace
        \begin{table}[h]
        \centering
        \begin{tabular}{|c|c|c|c|}
            \hline
            \textbf{Progetto} & \textbf{Analizzati} & \textbf{Veri positivi} & \textbf{Falsi positivi} \\
            \hline
            Accumulo & 50 & 37 & 13 \\
            Beam & 17 & 11 & 6 \\
            Bookkeeper & 50 & 38 & 12 \\
            Cassandra & 50 & 26 & 24 \\
            Druid & 50 & 41 & 9 \\
            Flink & 47 & 38 & 9 \\
            Geode & 50 & 35 & 15 \\
            Kafka & 50 & 43 & 7 \\
            Skywalking & 50 & 40 & 10 \\
            Zookeeper & 38 & 21 & 17 \\
            \hline
            Totale & 452 & 330 & 122 \\
            \hline
        \end{tabular}
        \caption{Risultati validazione Unnecessary Abstraction}
        \label{tab:caption} 
        \end{table}
    
    
    \newpage
    \subsection{Confronto dei risultati ottenuti con il tool Designite}
        Questa sezione propone un confronto tra i \textit{tool} Arcan e Designite \cite{Designite} , attraverso la \textit{detection} degli \textit{smell} \textit{SR, UUA e UNA} sui progetti Apache \cite{apache} presentati nella sezione 5.1.
% Questa sezione presenta un confronto riguardante i risultati ottenuti dall'analisi dei dieci progetti Apache \cite{apache} dalla detection effettuata con due tool differenti, Arcan e Designite \cite{Designite} (già introdotto nella sezione 2 \textit{Lavori correlati}). 

Il confronto non affronterà solamente la differenza tra la detection tra i due \textit{smell} in termini di istanze trovate, ma verranno effettuate anche analisi delle strategie di riconoscimento adottate dai due tool. Le strategie e gli algoritmi di riconoscimento sono stati esaminati attraverso ispezione del codice sorgente di Designite disponibile sulla piattaforma GitHub \cite{designiteGithub}.

È stato selezionato il tool Designite in merito a due considerazioni fondamentali:
%Per la selezione del tool da confrontare con il lavoro svolto su Arcan, la scelta è ricaduta su Designite per due considerazioni fondamentali:
\begin{enumerate}
    \item È in grado di effettuare la detection di \textit{Unutilizied Abstraction}, \textit{Unnecessary Abstraction} e di \textit{Broken Hierarchy (BH)}, uno \textit{smell} analogo a \textit{Subclasses Do Not Redefine Methods} citato come alias di \textit{BH} da G. Suryanarayana \cite{SURYANARAYANA201521}.
    
    \item La sua natura \textit{open-source} permette l'analisi del codice in maniera semplice e veloce.
\end{enumerate}\\
Per il confronto è stata preferita la presentazione degli algoritmi singolarmente, al fine di evidenziare in maniera chiara la discrepanza tra le strategie adottate e i valori ottenuti. La tabella 10 mette in risalto le diversità riscontrate nella detection dei progetti analizzati. Per lo \textit{smell} \textit{Unnecessary Abstraction} è indicato tra parentesi il numero di \textit{constant placeholder} trovati poiché Designite ricerca solamente questo particolare caso. 
%
\defaultvspace
\begin{table}[h]
    \centering
    \begin{tabular}{|c|c|c|c|}
        \hline
        \textbf{Tool} & \textbf{N\textsuperscript{o} SR} & \textbf{N\textsuperscript{o} UUA} & \textbf{N\textsuperscript{o} UNA}  \\
        \hline
        Arcan & 708 & 2033 & 2491 (178) \\
        Designite & 1623 & 6153 & 440 \\
        \hline
    \end{tabular}
    \caption{Differenza in numero assoluto per smell}
    \label{tab:caption}
\end{table}
\defaultvspace
\\
Come mostrato dai dati presentati dalla tabella 10, il \textit{tool} Designite è in grado di trovare un numero maggiore di istanze rispetto ad Arcan in merito a \textit{Subclasses Do Not Redefine Methods} e \textit{Unutilizied Abstraction}, rispettivamente con una differenza di 915 e 4120. Per quanto riguarda \textit{Unnecessary Abstraction} è Arcan a trovare un numero di \textit{smell} maggiore con 2051 istanze in più ma, limitando solamente il confronto al caso \textit{constant placeholder}, il numero di trovato da Designite è maggiore di 262 unità.

Il confronto tra questi \textit{smell} verrà approfondito nei tre paragrafi successivi. L'analisi è stata svolta seguendo la stessa struttura: a una piccola introduzione seguono considerazioni e differenze tra gli algoritmi di \textit{detection}, per concludere con la valutazione dei risultati ottenuti.

%Per quanto riguarda i risultati però va fatta una precisazione: in molti casi anche la struttura dati è fondamentale e ha un forte impatto sui FP considerati. Ovvero, Designite è specifico per Java mentre Arcan no, quindi alcuni costrutti sono trattati diversamente dai due parser e Des. ha info in più. Un'altro esempio è il grafo che non ha le classi Thread e risultano tutte smell ecc...


%Subclasses Do Not Redefine Methods
\subsubsection{Subclasses Do Not Redefine Methods} 
    Come anticipato in precedenza Designite è in grado di fare la \textit{detection} di \textit{Broken Hierarchy}, analogo allo \textit{smell} \textit{Subclasses Do Not Redefine Methods} e presentato anche come suo alias \cite{SURYANARAYANA201521}.
    
    \textit{Broken Hierarchy} si verifica quando un supertipo e il suo sottotipo non condividono una relazione IS-A con conseguente interruzione della sostituibilità. Lo \textit{smell} può presentarsi in tre differenti forme:
    \begin{itemize}
        \item I metodi del supertipo sono ancora applicabili o rilevanti nel sottotipo, sebbene non sia condivisa una relazione IS-A.
        
        \item Il sottotipo eredita funzioni del supertipo che non sono rilevanti o accettabili per i sottotipi.
        
        \item L'implementazione del sottotipo rifiuta esplicitamente metodi irrilevanti o inaccettabili ereditati dal supertipo.
    \end{itemize} \\
    \\
    %Differenze nelle definizioni
    Nonostante la descrizione di \textit{SR} e \textit{BH} possa sembrare differente, in tutte e due gli \textit{smell} non si verifica la condivisione della relazione IS-A da parte delle due classi e viene interrotta la sostituibilità. 
    
    %Differenza strategia identificazione
    \paragraph{Strategie di identificazione}
        Il controllo effettuato da Designite per la ricerca di \textit{Broken Hierarchy} è analogo a quello effettuato da Arcan, poiché viene effettuata una ricerca di tutte le classi che non condividono almeno un metodo con i loro supertipi. Di seguito è illustrato l'algoritmo utilizzato da Designite per la \textit{detection} di Broken Hierarchy:
        \defaultvspace
        \begin{algorithmic}
            \Function{designite-bh-detector}{ }
                \If{Il \textit{type} analizzato ha \textit{supertypes} e almeno un metodo pubblico}
                    \For{Ogni \textit{supertype} del \textit{type} analizzato}
                        \If{\textit{type} e \textit{supertype} non condividono almeno un metodo}
                            \State{\textbf{return} l'elemento è una Broken Hierarchy}
                        \EndIf
                    \EndFor
                \EndIf
            \EndFunction
        \end{algorithmic}
        \defaultvspace
        Si può notare una differenza tra i due algoritmi nella ricerca dei metodi, poiché Designite è in grado di effettuare una loro distinzione per modificatore (\textit{public, private} oppure \textit{protected}) e valuta solamente i metodi effettivamente ereditati da una classe, mentre Arcan considera tutte le funzioni definite dai supertipi indipendentemente dalla loro visibilità.
        Alcune analisi riguardanti il funzionamento e la struttura del codice di Designite hanno mostrato che per il controllo di \textit{Broken Hierarchy} e in particolare per verificare le condivisione di un metodo da parte di due classi viene utilizzato il nome della funzione e non la \textit{signature}. Anche Designite quindi considera i casi di \textit{overloading} come ridefinizione del metodo, analogamente a quanto è stato deciso per la \textit{detection} in Arcan.
        
    \paragraph{Analisi risultati}
        Nonostante le analogie tra le strategie e gli algoritmi di \textit{detection}, il numero di istanze trovate da Designite (1623) è maggiore rispetto ad Arcan (708), con un divario di 915 unità. È complicato effettuare una ricerca delle cause di questa discrepanza ma una di queste può essere identificata con la differenza nella scelta delle classi da analizzare, poiché Arcan effettua un filtraggio delle \textit{abstraction} rimuovendo tutte le interfacce e le classi di \textit{exception} che Designite invece esamina (si tratta di 286 istanze). 
        
        Arcan potrebbe inoltre aver valutato come non affette dallo \textit{smell} diverse \textit{abstraction} che condividono il nome del metodo con una funzione definita nella gerarchia del padre ma dotata di modificatore \textit{private}, non ereditabile quindi dai suoi sottotipi. Designite invece, essendo in grado di discriminare le funzioni ereditate in base al modificatore, non soffre di questo problema. %Per fare un esempio, si può immaginare la presenza di una gerarchia tra tre classi A, B e C, con la classe A che è il primo supertipo della gerarchia e C l'ultimo sottotipo e con le classi A e C che condividono entrambe un metodo \textit{private} chiamato "\textit{bar}".
        %Mentre Arcan non riconoscerebbe questa situazione come smell, in quanto nella gerarchia due metodi condividono lo stesso nome, lo stesso verrebbe correttamente segnalato da Designite poiché è in grado di discriminare le funzioni ereditate in base ai loro modificatori e, essendo il metodo nella classe A \textit{private}, non verrebbe ereditato da B e quindi non ci sarebbe ridefinizione da parte di C.
        
        %Un'ulteriore confronto tra le classi validate trovate da Arcan e i risultati di Designite non ha portato alcun risultato significativo. 
        %L'unica informazione ricevuta da questa analisi è il fatto che Designite non sembra soffrire dello stesso problema di Arcan con le classi derivate dalle classi di sistema che non sono presenti nel grafo.
        % Non si vede nessun pattern specifico, alcune classi affette da un probelma (es. la storia delle interfacce del padre astratto) vengono riconosciute come smell da Designite mentre altre affette dallo stesso problema no (in Arcan prima erano considerate FP e dopo le modifiche sono sparite dalla detection).
        % La diversità nel parsing e nelle strutture dati utilizzate, che può portare inevitabilmente a risultati diversi (poichè ogni strategia adottata e ogni struttura ha i suoi problemi e le sue debolezze)
        \defaultvspace
        \begin{table}[h]
            \centering
            \begin{tabular}{|c|c|c|}
                \hline
                \textbf{Progetto} & \textbf{N\textsuperscript{o} smell Arcan} & \textbf{N\textsuperscript{o} smell Designite} \\
                \hline
                Accumulo & 98 & 71 \\
                Beam & 5 & 5 \\
                Bookkeeper & 52 & 149\\
                Cassandra & 96 & 247\\
                Druid & 6 & 86\\
                Flink & 66 & 112\\
                Geode & 143 & 533\\
                Kafka & 138 & 243\\
                Skywalking & 65 & 109\\
                Zookeeper & 39 & 68\\
                \hline
                Totale & 708 & 1623 \\
                \hline
            \end{tabular}
            \caption{Confronto istanze Subclasses Do Not Redefine Methods}
            \label{tab:caption}
        \end{table}
        \defaultvspace


%Unutilizied Abstraction

\subsubsection{Unutilizied Abstraction}
    Designite è in grado di effettuare la \textit{detection} anche dello \textit{smell} \textit{Unutilizied Abstraction}, rendendo così possibile il confronto con Arcan. I due \textit{tool} presentano strategie di identificazione leggermente differenti, che causano una discrepanza sostanziale nel numero di istanze identificate con 4120 unità in più riscontrate da Designite.

    %Differenza strategia identificazione
    \paragraph{Strategie di identificazione}
        La \textit{detection} di Unutilizied Abstraction da parte del \textit{tool} Designite comporta piccole differenze con le strategie di identificazione individuate per Arcan. L'algoritmo definito da Designite per la \textit{detection} dello \textit{smell} è stato definito come segue:
        \defaultvspace
        \begin{algorithmic}
            \Function{designite-uua-detector}{ }
                \If{il tipo analizzato non ha dipendenze in ingresso}
                     \State{\textbf{return} l'elemento è una Unutilizied Abstraction}
                \Else \If{L'elemento analizzato ha supertipi}
                    \If{Tutti i supertipi non hanno dipendenze in ingresso}
                        \State{\textbf{return} l'elemento è una Unutilizied Abstraction}
                    \EndIf
                \EndIf \EndIf
            \EndFunction
        \end{algorithmic}
        \defaultvspace
        %
        Si può notare che non vengono valutati separatamente i due casi \textit{Unreferenced Abstraction} e \textit{Orphan Abstraction} (presentati nella sezione 4.2) a differenza di Arcan che invece effettua controlli differenti a seconda della tipologia di \textit{abstraction} considerata.
        Non essendoci nessuna distinzione tra tipologia della classe e delle relazioni tra esse, vengono considerati come \textit{smell} tutte le \textit{abstraction} che non presentano dipendenze in ingresso. 
        %Questa differenza può essere data da due motivi in particolare:
        %Sono state formulate due ipotesi differenti per giustificare questa differenza nella detection:
        %\begin{itemize}
        %    \item Non effettuare la distinzione è stata una scelta implementativa degli sviluppatori. È possibile che sia stata dettata anche da un impossibilità ad effettuare controlli separati a causa del parser e della struttura dati utilizzati.
            
        %    \item il parsing effettuato da Designite e la sua architettura permettono di avere solamente dipendenze di un certo tipo per certe classi, perciò un interfaccia senza dipendenze significa che non è stata implementata da nessuno (a differenza di Arcan dove vengono rilevati diversi tipi di nodi, ad esempio dependsOn, anche per le interfacce).
        %\end{itemize}
        %\\
        Viene effettuato inoltre un altro controllo per stabilire se una classe è inutilizzata o meno, attraverso una valutazione dei supertipi della classe sotto analisi. Se la \textit{abstraction} che stiamo considerando presenta dei supertipi, la presenza dello \textit{smell} è verificata, oltre che dalla eventuale assenza di dipendenze in ingresso, anche dalle dipendenze dei supertipi stessi. Se nessuno di essi ne presenta in ingresso allora anche la classe sotto analisi è considerata \textit{unutilizied}. Più in generale si può affermare che Designite considera come tali tutte le \textit{abstraction} figlie di classi senza dipendenze in ingresso. 
        
    % Risultati
    \paragraph{Analisi risultati}
        Nella tabella 12 vengono riportati i dati relativi alla \textit{detection} effettuata con i due diversi \textit{tool}. Da questi si evince che le istanze dello \textit{smell} trovate da Designite sono in numero maggiore rispetto ad Arcan, con una differenza di 4120 unità. Un'analisi approfondita del funzionamento dei due \textit{tool} ha dimostrato che la causa principale di questo divario può essere ricercata nel \textit{parsing} degli elementi e nelle strutture dati utilizzate. La verifica è stata effettuata selezionando le \textit{abstraction} individuate come \textit{smell} da Designite ma non considerate tali da parte di Arcan, e analizzando successivamente la struttura del grafo delle dipendenze al fine di valutare le relazioni della classe dal punto di vista di Arcan.
        È stata riscontrata la presenza di diverse classi nel grafo con archi in ingresso e nessun supertipo, perciò a causa delle relazioni presenti non sono state individuati da Arcan come \textit{unutilizied} e la loro rilevazione da parte di Designite non è causata del controllo sui supertipi. Di conseguenza si può concludere che con molta probabilità sia presente una differenza tra le strutture dati e gli algoritmi di \textit{parsing} utilizzati, con le dipendenze che vengono identificate in maniera diversa causando un grande divario di istanze dello \textit{smell} tra i due \textit{tool}.
        
        Inoltre un altro fattore che potrebbe influenzare il numero delle istanze è la considerazione come \textit{smell} da parte di Designite di tutti i figli di classi inutilizzate indistintamente dal loro utilizzo o meno. Se queste ultime fossero utilizzate, non verrebbero infatti considerate da Arcan come problema, creando una discrepanza nei risultati. 

    
        \defaultvspace
        \begin{table}[h]
            \centering
            \begin{tabular}{|c|c|c|}
                \hline
                \textbf{Progetto} & \textbf{N\textsuperscript{o} smell Arcan} & \textbf{N\textsuperscript{o} smell Designite} \\
                \hline
                Accumulo & 177 & 1955 \\
                Beam & 32 & 22\\
                Bookkeeper & 189 & 906\\
                Cassandra & 137 & 1927\\
                Druid & 128 & 203\\
                Flink & 185 & 431\\
                Geode & 304 & 1557\\
                Kafka & 134 & 1118\\
                Skywalking & 683 & 1182\\
                Zookeeper & 64 & 336\\
                \hline
                Totale & 2033 & 6153\\
                \hline
            \end{tabular}
            \caption{Confronto istanze Unutilizied Abstraction}
            \label{tab:caption}
        \end{table}

%Unnecessary Abstraction

\subsubsection{Unnecessary Abstraction}
    Attraverso il \textit{tool} Designite è possibile anche la \textit{detection} dello \textit{smell} \textit{Unnecessary Abstraction}, sebbene con la limitazione di identificare solamente il caso \textit{constant placeholder}, a differenza di Arcan che individua anche \textit{procedural class} e le \textit{over engineered}. Questa diversità tra le capacità dei due \textit{smell} crea un grande divario anche per quanto riguarda il numero di istanze, con Arcan che ne ha individuate 2051 in più.
    
    Il confronto delle strategie di identificazione sarà effettuato analizzando solamente il caso \textit{constant placeholder} mentre l'analisi dei risultati verrà condotta esaminando anche tutti i casi presenti. Nello specifico, saranno inizialmente considerati tutti i casi di \textit{Unnecessary Abstraction}, per poi proseguire con l'analisi solamente di \textit{constant placeholder}.
    
    
    \paragraph{Strategie di identificazione}
        Per la ricerca delle \textit{abstraction} \textit{constant placeholder} gli sviluppatori di Designite hanno effettuato una ricerca di tutte le classi senza metodi e con un numero di attributi minore o uguale a una determinata soglia, che al momento della stesura di questo elaborato assumeva il valore 5. 
        Segue una rappresentazione in pseudocodice dell'algoritmo di \textit{detection}:
        \defaultvspace
        \begin{algorithmic}
            \Function{designite-una-detector}{ }
                \State{def SOGLIA = 5} 
                \If{L'elemento analizzato ha un numero di attributi $<=$ SOGLIA}
                    \If{L'elemento analizzato non ha metodi}
                        \State{\textbf{return} l'elemento è una Unnecessary Abstraction}
                    \EndIf
                \EndIf
            \EndFunction
        \end{algorithmic}
        \defaultvspace \\
        %Gli algoritmi di ricerca implementati dai due tool presentano alcune differenze.
        %Oltre alla differenza degli algoritmi di ricerca tra le casistiche considerate, di cui è stato già discusso in precedenza, la sola \textit{detection} del caso \textit{constant placeholder} presenta ulteriori diversità. 
        \\
        Mentre Arcan considera come \textit{constant placeholder} le \textit{abstraction} che presentano solamente attributi costanti a prescindere dal loro numero, Designite introduce una soglia minima di attributi che deve definire una classe per essere considerata una \textit{constant placeholder}, poiché gli sviluppatori del tool devono aver ritenuto che classi con pochi costante non possano essere considerati \textit{placeholder}. Il problema delle classi costanti con pochi attributi è stato riscontrato in Arcan, dove è emerso che la causa principale della presenza di falsi positivi nel caso \textit{constant placeholder} fosse proprio la loro presenza nel sistema.
        
        Un ulteriore differenza tra i due algoritmi si può ricercare nel controllo degli attributi poiché, mentre Arcan verifica che gli siano costanti e con un valore già assegnato, Designite si limita solamente a verificare che la \textit{abstraction} considerata non contenga altro oltre ad essi, effettuando così la \textit{detection} di classi che non hanno tutti valori costanti.
         
         Infine la struttura di Arcan ha reso necessario anche il filtraggio delle classi rappresentanti errori, mentre per Designite non è stato necessario poiché probabilmente l'introduzione della soglia esclude queste classi che solitamente contengono massimo due attributi costanti.
    
    \paragraph{Analisi risultati}
        Essendo condotta l'analisi nei due modi differenti, nella tabella riepilogativa delle istanze è stata inserita una nuova colonna, denominata \textit{N\textsuperscript{o} constant placeholder}, che indica il numero di istanze di questa tipologia trovate da Arcan nei vari progetti.
        
        Dai dati presentati nella tabella 13 emergono i diversi approcci adottati dai due tool. La differenza di 2051 istanze in favore di Arcan è in soprattutto conseguenza della ricerca delle \textit{procedural class} che ha un forte impatto sul numero totale di \textit{smell} trovati. \\
        
        \defaultvspace
        \begin{table}[h]
            \centering
            \begin{tabular}{|c|c|c|c|}
                \hline
                \textbf{Progetto} & \textbf{N\textsuperscript{o} smell} & \textbf{N\textsuperscript{o} constant} & \textbf{N\textsuperscript{o} smell} \\
                & \textbf{Arcan} & \textbf{placeholder} & \textbf{designite}\\
                \hline
                Accumulo & 1113 & 24 & 77 \\
                Beam & 13 & 24 & 2\\
                Bookkeeper & 96 & 15 & 74\\
                Cassandra & 639 & 4 & 33\\
                Druid & 52 & 5 & 7\\
                Flink & 47 & 25 & 28\\
                Geode & 166 & 19 & 34\\
                Kafka & 119 & 19 & 51\\
                Skywalking & 208 & 36 & 128\\
                Zookeeper & 38 & 7 & 6\\
                \hline
                Totale & 2491 & 178 & 440 \\
                \hline
            \end{tabular}
            \caption{Confronto istanze Unnecessary Abstraction}
            \label{tab:caption}
        \end{table}
        Se anche per Arcan non venissero considerati i due casi \textit{over engineered} e \textit{procedural class}, si può notare come Designite sia in grado di trovare un numero maggiore di istanze. 
            %Questa differenza può essere riflesso del fatto che la ricerca di Arcan effettua controlli più approfonditi sugli attributi.
            %Questa differenza può essere un riflesso del fatto che, come analizzato nel paragrafo precedente, nella ricerca Designite non effettua il controllo che tutti gli attributi della classe siano valori già assegnati e costanti, e per questo motivo un gran numero di classi escluse da Arcan sono considerate come smell invece da Designite. Anche per questo smell è stata effettuata un'analisi delle classi trovate da Designite, al fine di enfatizzare meglio le differenze. Questa analisi ha confermato quello detto in precedenza, e cioè che non viene effettuato nessun controllo sugli attributi finale. Inoltre ha evidenziato un'altra differenza fondamentale tra i due tool: mentre Arcan controlla solamente gli attributi definiti dalla classe sotto esame, Designite considera anche tutti quelli ereditati da eventuali supertipi. Questo causa quindi la presenza di molte classi vuote che, essendo sottotipi di \textit{unnecessary smell}, sono anch'esse affette dallo smell; in Arcan questo non succederebbe. Inoltre non vengono considerati i valori di default
        %Alternative
        Un'analisi dettagliata delle classi individuate da quest'ultimo ha permesso di evidenziare le possibili cause principali che hanno portato a questa discrepanza:
            %Un'analisi delle classi trovate da Designite ha permesso di evedenziare due delle cause che hanno portato a questa differenza tra i due tool:
        \begin{itemize}
            \item Il \textit{tool} Designite non effettua il controllo che tutti gli attributi della classe siano costanti e con valori di default assegnati e l'analisi effettuata ha confermato questa differenza, in quanto tra le istanze erano presenti numerose classi che presentavano attributi non costanti.
            
            \item C'è una differenza per quanto riguarda gli attributi considerati all'interno delle classi, poiché mentre Arcan effettua l'analisi solamente degli attributi della \textit{abstraction} presa sotto analisi, Designite considera anche tutti quelli ereditati da eventuali supertipi. Questa diversità porta quindi la \textit{detection} da parte di Designite di classi vuote che sono sottotipi di \textit{costant placeholder}, che non verrebbero invece individuate da Arcan.
        \end{itemize}
        
        

%Conclusioni
\subsubsection{Conclusioni}
La sezione 5.4 ha illustrato un confronto tra i due \textit{tool} Arcan e Designite, in grado di effettuare la \textit{detection} degli smell \textit{Subclasses Do Not Redefine Methods}, \textit{Unutilizied Abstraction} e \textit{Unnecessary Abstraction}.

Designite ha dimostrato una capacità maggiore di riconoscimento di istanze per gli smell \textit{SR} e \textit{UUA} ma poi ne ha individuato un numero minore per quanto riguarda lo smell \textit{UNA}. Non essendo presenti i valori di \textit{precision} per Designite, non è stato possibile fare un confronto approfondito riguardo questo aspetto degli algoritmi. In generale, è stata constatata sia la presenza di diversi falsi positivi di Arcan non segnalati da Designite sia istanze individuate da Designite che non sono risultate come smell.

Riguardo \textit{Subclasses Do Not Redefine Methods}, Designite è in grado di discriminare i metodi ereditati in base al modificatore mentre Arcan li considera tutti, anche se ritengo un caso abbastanza isolato la presenza in una gerarchia di due metodi \textit{private} definiti da classi diverse che condividono lo stesso nome. Arcan inoltre non considera nella ricerca diversi falsi positivi relativi a \textit{exception}, riconosciuti invece da Designite. Per questi motivi introdotti, ritengo che nessun \textit{tool} prevalga sull'altro in considerazione alla qualità e capacità della \textit{detection}.
    %Per questi motivi introdotti, ritengo che la \textit{detection} di Arcan per questo smell sia più accurata, anche a fronte di una \textit{precision} di 89.75 \%.

In merito a \textit{Unutilizied Abstraction}, Designite non effettua alcuna distinzione tra la tipologia di classi analizzate e le relazioni che intercorrono tra esse, a differenza di quanto effettuato da Arcan. Questa situazione porta quindi a non considerare il caso delle \textit{orphan abstractions}, introdotto nella definizione dello smell come una delle due tipologie di manifestazione (sezione 4.2). Molte interfacce e classi astratte, che dovrebbero risultare come smell, non vengono erroneamente considerate da Designite.
%Questo causa la detection anche di interfacce e classi astratte non implementate oppure estese, come richiede lo smell, ma soltanto utilizzate attraverso la definizione di attributi di quel tipo oppure l'utilizzo di metodi e costanti definite al loro interno.
Inoltre la ricerca dei figli di classi inutilizzate introduce diverse istanze in più non prese in considerazione da Arcan. Il controllo di Arcan sembra quindi più completo, a fronte di un'analisi maggiormente approfondita su classi e relazioni e una fedeltà superiore alla definizione dello smell rispetto a Designite.
    % La loro mi sembra sbagliata, la ns. ha una precision alta 95.43 \%

Infine, relativamente a \textit{Unnecessary Abstraction} si può affermare che Arcan effettua un controllo più dettagliato e accurato, poiché vengono ricercati nel codice tutti e tre i casi di classi non necessarie presentati, sacrificando però la \textit{precision} (73.66\%) influenzata soprattutto dalle classi procedurali non considerate da Designite. Anche valutando solamente il caso \textit{constant placeholder}, Arcan esegue verifiche più approfondite riguardo le \textit{abstraction} sotto analisi.

Concludendo, si può affermare che nonostante la ricerca di Designite sia in grado di trovare un maggior numero di istanze, Arcan effettua controlli più dettagliati sulle classi e interfacce sotto analisi. Non disponendo però dei dati di \textit{precision} di tutte e due i \textit{tool}, non può essere effettuato un confronto sulla differenza della reale efficacia dei due algoritmi.




% Non sempre più vuol dire meglio, di designite non abbiamo i valori di precision

% Designite ne trova di più
% Designite considera alcuni FP di Arcan ma Arcan considera alcuni casi come FP di cui Designite non effettua la detection

    
    \subsection{Problemi riscontrati}
        Durante lo sviluppo delle nuove strutture e algoritmi di \textit{detection} sono stati affrontati diversi problemi, riguardanti sia lo studio di soluzioni adeguate ai problemi sia l'effettiva implementazione e correzione degli errori. 

L'attività di validazione del lavoro svolto è stata l'attività più complessa e impegnativa poiché i controlli effettuati sui differenti progetti hanno comportato modifiche alle strutture e algoritmi implementati in precedenza. L'analisi approfondita delle diverse classi individuate ha rivelato la presenza di diversi falsi positivi, causati da casi particolari che non erano stati considerati e da situazioni che non era ben chiaro se fossero incluse nelle definizioni degli \textit{smell} o meno. 
Per alcuni di questi casi particolari è stata sufficiente la modifica degli algoritmi di \textit{detection} e \textit{parsing} oppure della struttura del grafo delle dipendenze, per migliorare la precisione e filtrare i casi positivi. Per la soluzione di altre situazioni invece è stato necessario, prima di procedere con un'eventuale modifica del software, un confronto con la mia tutor, per comprendere quale fosse il metodo migliore per la loro gestione.

Per gli \textit{smell} \textit{Subclasses Do Not Redefine Methods} e \textit{Unnecessary Abstraction} le modifiche apportate hanno inoltre comportato lo svolgimento di una seconda fase di validazione, per definire i nuovi valori di \textit{precision} degli algoritmi.




        
    \subsection{Osservazioni finali}
        In conclusione, rispetto agli \textit{architectural smell} di cui è stata sviluppata la \textit{detection} e dopo aver eseguito Arcan su dieci differenti sistemi \textit{open source}, Arcan presenta una \textit{precision} riguardo \textit{Subclasses Do Not Redefine Methods, Unutilizied Abstraction e Unnecessary Abstraction} rispettivamente del 89.75\%, 95.43\% e 73.66\%.
    