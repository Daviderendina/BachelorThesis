\subsubsection{Conclusioni}
La sezione 5.4 ha illustrato un confronto tra i due \textit{tool} Arcan e Designite, in grado di effettuare la \textit{detection} degli smell \textit{Subclasses Do Not Redefine Methods}, \textit{Unutilizied Abstraction} e \textit{Unnecessary Abstraction}.

Designite ha dimostrato una capacità maggiore di riconoscimento di istanze per gli smell \textit{SR} e \textit{UUA} ma poi ne ha individuato un numero minore per quanto riguarda lo smell \textit{UNA}. Non essendo presenti i valori di \textit{precision} per Designite, non è stato possibile fare un confronto approfondito riguardo questo aspetto degli algoritmi. In generale, è stata constatata sia la presenza di diversi falsi positivi di Arcan non segnalati da Designite sia istanze individuate da Designite che non sono risultate come smell.

Riguardo \textit{Subclasses Do Not Redefine Methods}, Designite è in grado di discriminare i metodi ereditati in base al modificatore mentre Arcan li considera tutti, anche se ritengo un caso abbastanza isolato la presenza in una gerarchia di due metodi \textit{private} definiti da classi diverse che condividono lo stesso nome. Arcan inoltre non considera nella ricerca diversi falsi positivi relativi a \textit{exception}, riconosciuti invece da Designite. Per questi motivi introdotti, ritengo che nessun \textit{tool} prevalga sull'altro in considerazione alla qualità e capacità della \textit{detection}.
    %Per questi motivi introdotti, ritengo che la \textit{detection} di Arcan per questo smell sia più accurata, anche a fronte di una \textit{precision} di 89.75 \%.

In merito a \textit{Unutilizied Abstraction}, Designite non effettua alcuna distinzione tra la tipologia di classi analizzate e le relazioni che intercorrono tra esse, a differenza di quanto effettuato da Arcan. Questa situazione porta quindi a non considerare il caso delle \textit{orphan abstractions}, introdotto nella definizione dello smell come una delle due tipologie di manifestazione (sezione 4.2). Molte interfacce e classi astratte, che dovrebbero risultare come smell, non vengono erroneamente considerate da Designite.
%Questo causa la detection anche di interfacce e classi astratte non implementate oppure estese, come richiede lo smell, ma soltanto utilizzate attraverso la definizione di attributi di quel tipo oppure l'utilizzo di metodi e costanti definite al loro interno.
Inoltre la ricerca dei figli di classi inutilizzate introduce diverse istanze in più non prese in considerazione da Arcan. Il controllo di Arcan sembra quindi più completo, a fronte di un'analisi maggiormente approfondita su classi e relazioni e una fedeltà superiore alla definizione dello smell rispetto a Designite.
    % La loro mi sembra sbagliata, la ns. ha una precision alta 95.43 \%

Infine, relativamente a \textit{Unnecessary Abstraction} si può affermare che Arcan effettua un controllo più dettagliato e accurato, poiché vengono ricercati nel codice tutti e tre i casi di classi non necessarie presentati, sacrificando però la \textit{precision} (73.66\%) influenzata soprattutto dalle classi procedurali non considerate da Designite. Anche valutando solamente il caso \textit{constant placeholder}, Arcan esegue verifiche più approfondite riguardo le \textit{abstraction} sotto analisi.

Concludendo, si può affermare che nonostante la ricerca di Designite sia in grado di trovare un maggior numero di istanze, Arcan effettua controlli più dettagliati sulle classi e interfacce sotto analisi. Non disponendo però dei dati di \textit{precision} di tutte e due i \textit{tool}, non può essere effettuato un confronto sulla differenza della reale efficacia dei due algoritmi.




% Non sempre più vuol dire meglio, di designite non abbiamo i valori di precision

% Designite ne trova di più
% Designite considera alcuni FP di Arcan ma Arcan considera alcuni casi come FP di cui Designite non effettua la detection
