Durante lo sviluppo delle nuove strutture e algoritmi di \textit{detection} sono stati affrontati diversi problemi, riguardanti sia lo studio di soluzioni adeguate ai problemi sia l'effettiva implementazione e correzione degli errori. 

L'attività di validazione del lavoro svolto è stata l'attività più complessa e impegnativa poiché i controlli effettuati sui differenti progetti hanno comportato modifiche alle strutture e algoritmi implementati in precedenza. L'analisi approfondita delle diverse classi individuate ha rivelato la presenza di diversi falsi positivi, causati da casi particolari che non erano stati considerati e da situazioni che non era ben chiaro se fossero incluse nelle definizioni degli \textit{smell} o meno. 
Per alcuni di questi casi particolari è stata sufficiente la modifica degli algoritmi di \textit{detection} e \textit{parsing} oppure della struttura del grafo delle dipendenze, per migliorare la precisione e filtrare i casi positivi. Per la soluzione di altre situazioni invece è stato necessario, prima di procedere con un'eventuale modifica del software, un confronto con la mia tutor, per comprendere quale fosse il metodo migliore per la loro gestione.

Per gli \textit{smell} \textit{Subclasses Do Not Redefine Methods} e \textit{Unnecessary Abstraction} le modifiche apportate hanno inoltre comportato lo svolgimento di una seconda fase di validazione, per definire i nuovi valori di \textit{precision} degli algoritmi.



