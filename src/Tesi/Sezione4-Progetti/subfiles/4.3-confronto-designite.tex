Questa sezione propone un confronto tra i \textit{tool} Arcan e Designite \cite{Designite} , attraverso la \textit{detection} degli \textit{smell} \textit{SR, UUA e UNA} sui progetti Apache \cite{apache} presentati nella sezione 5.1.
% Questa sezione presenta un confronto riguardante i risultati ottenuti dall'analisi dei dieci progetti Apache \cite{apache} dalla detection effettuata con due tool differenti, Arcan e Designite \cite{Designite} (già introdotto nella sezione 2 \textit{Lavori correlati}). 

Il confronto non affronterà solamente la differenza tra la detection tra i due \textit{smell} in termini di istanze trovate, ma verranno effettuate anche analisi delle strategie di riconoscimento adottate dai due tool. Le strategie e gli algoritmi di riconoscimento sono stati esaminati attraverso ispezione del codice sorgente di Designite disponibile sulla piattaforma GitHub \cite{designiteGithub}.

È stato selezionato il tool Designite in merito a due considerazioni fondamentali:
%Per la selezione del tool da confrontare con il lavoro svolto su Arcan, la scelta è ricaduta su Designite per due considerazioni fondamentali:
\begin{enumerate}
    \item È in grado di effettuare la detection di \textit{Unutilizied Abstraction}, \textit{Unnecessary Abstraction} e di \textit{Broken Hierarchy (BH)}, uno \textit{smell} analogo a \textit{Subclasses Do Not Redefine Methods} citato come alias di \textit{BH} da G. Suryanarayana \cite{SURYANARAYANA201521}.
    
    \item La sua natura \textit{open-source} permette l'analisi del codice in maniera semplice e veloce.
\end{enumerate}\\
Per il confronto è stata preferita la presentazione degli algoritmi singolarmente, al fine di evidenziare in maniera chiara la discrepanza tra le strategie adottate e i valori ottenuti. La tabella 10 mette in risalto le diversità riscontrate nella detection dei progetti analizzati. Per lo \textit{smell} \textit{Unnecessary Abstraction} è indicato tra parentesi il numero di \textit{constant placeholder} trovati poiché Designite ricerca solamente questo particolare caso. 
%
\defaultvspace
\begin{table}[h]
    \centering
    \begin{tabular}{|c|c|c|c|}
        \hline
        \textbf{Tool} & \textbf{N\textsuperscript{o} SR} & \textbf{N\textsuperscript{o} UUA} & \textbf{N\textsuperscript{o} UNA}  \\
        \hline
        Arcan & 708 & 2033 & 2491 (178) \\
        Designite & 1623 & 6153 & 440 \\
        \hline
    \end{tabular}
    \caption{Differenza in numero assoluto per smell}
    \label{tab:caption}
\end{table}
\defaultvspace
\\
Come mostrato dai dati presentati dalla tabella 10, il \textit{tool} Designite è in grado di trovare un numero maggiore di istanze rispetto ad Arcan in merito a \textit{Subclasses Do Not Redefine Methods} e \textit{Unutilizied Abstraction}, rispettivamente con una differenza di 915 e 4120. Per quanto riguarda \textit{Unnecessary Abstraction} è Arcan a trovare un numero di \textit{smell} maggiore con 2051 istanze in più ma, limitando solamente il confronto al caso \textit{constant placeholder}, il numero di trovato da Designite è maggiore di 262 unità.

Il confronto tra questi \textit{smell} verrà approfondito nei tre paragrafi successivi. L'analisi è stata svolta seguendo la stessa struttura: a una piccola introduzione seguono considerazioni e differenze tra gli algoritmi di \textit{detection}, per concludere con la valutazione dei risultati ottenuti.

%Per quanto riguarda i risultati però va fatta una precisazione: in molti casi anche la struttura dati è fondamentale e ha un forte impatto sui FP considerati. Ovvero, Designite è specifico per Java mentre Arcan no, quindi alcuni costrutti sono trattati diversamente dai due parser e Des. ha info in più. Un'altro esempio è il grafo che non ha le classi Thread e risultano tutte smell ecc...


%Subclasses Do Not Redefine Methods
\subsubsection{Subclasses Do Not Redefine Methods} 
    Come anticipato in precedenza Designite è in grado di fare la \textit{detection} di \textit{Broken Hierarchy}, analogo allo \textit{smell} \textit{Subclasses Do Not Redefine Methods} e presentato anche come suo alias \cite{SURYANARAYANA201521}.
    
    \textit{Broken Hierarchy} si verifica quando un supertipo e il suo sottotipo non condividono una relazione IS-A con conseguente interruzione della sostituibilità. Lo \textit{smell} può presentarsi in tre differenti forme:
    \begin{itemize}
        \item I metodi del supertipo sono ancora applicabili o rilevanti nel sottotipo, sebbene non sia condivisa una relazione IS-A.
        
        \item Il sottotipo eredita funzioni del supertipo che non sono rilevanti o accettabili per i sottotipi.
        
        \item L'implementazione del sottotipo rifiuta esplicitamente metodi irrilevanti o inaccettabili ereditati dal supertipo.
    \end{itemize} \\
    \\
    %Differenze nelle definizioni
    Nonostante la descrizione di \textit{SR} e \textit{BH} possa sembrare differente, in tutte e due gli \textit{smell} non si verifica la condivisione della relazione IS-A da parte delle due classi e viene interrotta la sostituibilità. 
    
    %Differenza strategia identificazione
    \paragraph{Strategie di identificazione}
        Il controllo effettuato da Designite per la ricerca di \textit{Broken Hierarchy} è analogo a quello effettuato da Arcan, poiché viene effettuata una ricerca di tutte le classi che non condividono almeno un metodo con i loro supertipi. Di seguito è illustrato l'algoritmo utilizzato da Designite per la \textit{detection} di Broken Hierarchy:
        \defaultvspace
        \begin{algorithmic}
            \Function{designite-bh-detector}{ }
                \If{Il \textit{type} analizzato ha \textit{supertypes} e almeno un metodo pubblico}
                    \For{Ogni \textit{supertype} del \textit{type} analizzato}
                        \If{\textit{type} e \textit{supertype} non condividono almeno un metodo}
                            \State{\textbf{return} l'elemento è una Broken Hierarchy}
                        \EndIf
                    \EndFor
                \EndIf
            \EndFunction
        \end{algorithmic}
        \defaultvspace
        Si può notare una differenza tra i due algoritmi nella ricerca dei metodi, poiché Designite è in grado di effettuare una loro distinzione per modificatore (\textit{public, private} oppure \textit{protected}) e valuta solamente i metodi effettivamente ereditati da una classe, mentre Arcan considera tutte le funzioni definite dai supertipi indipendentemente dalla loro visibilità.
        Alcune analisi riguardanti il funzionamento e la struttura del codice di Designite hanno mostrato che per il controllo di \textit{Broken Hierarchy} e in particolare per verificare le condivisione di un metodo da parte di due classi viene utilizzato il nome della funzione e non la \textit{signature}. Anche Designite quindi considera i casi di \textit{overloading} come ridefinizione del metodo, analogamente a quanto è stato deciso per la \textit{detection} in Arcan.
        
    \paragraph{Analisi risultati}
        Nonostante le analogie tra le strategie e gli algoritmi di \textit{detection}, il numero di istanze trovate da Designite (1623) è maggiore rispetto ad Arcan (708), con un divario di 915 unità. È complicato effettuare una ricerca delle cause di questa discrepanza ma una di queste può essere identificata con la differenza nella scelta delle classi da analizzare, poiché Arcan effettua un filtraggio delle \textit{abstraction} rimuovendo tutte le interfacce e le classi di \textit{exception} che Designite invece esamina (si tratta di 286 istanze). 
        
        Arcan potrebbe inoltre aver valutato come non affette dallo \textit{smell} diverse \textit{abstraction} che condividono il nome del metodo con una funzione definita nella gerarchia del padre ma dotata di modificatore \textit{private}, non ereditabile quindi dai suoi sottotipi. Designite invece, essendo in grado di discriminare le funzioni ereditate in base al modificatore, non soffre di questo problema. %Per fare un esempio, si può immaginare la presenza di una gerarchia tra tre classi A, B e C, con la classe A che è il primo supertipo della gerarchia e C l'ultimo sottotipo e con le classi A e C che condividono entrambe un metodo \textit{private} chiamato "\textit{bar}".
        %Mentre Arcan non riconoscerebbe questa situazione come smell, in quanto nella gerarchia due metodi condividono lo stesso nome, lo stesso verrebbe correttamente segnalato da Designite poiché è in grado di discriminare le funzioni ereditate in base ai loro modificatori e, essendo il metodo nella classe A \textit{private}, non verrebbe ereditato da B e quindi non ci sarebbe ridefinizione da parte di C.
        
        %Un'ulteriore confronto tra le classi validate trovate da Arcan e i risultati di Designite non ha portato alcun risultato significativo. 
        %L'unica informazione ricevuta da questa analisi è il fatto che Designite non sembra soffrire dello stesso problema di Arcan con le classi derivate dalle classi di sistema che non sono presenti nel grafo.
        % Non si vede nessun pattern specifico, alcune classi affette da un probelma (es. la storia delle interfacce del padre astratto) vengono riconosciute come smell da Designite mentre altre affette dallo stesso problema no (in Arcan prima erano considerate FP e dopo le modifiche sono sparite dalla detection).
        % La diversità nel parsing e nelle strutture dati utilizzate, che può portare inevitabilmente a risultati diversi (poichè ogni strategia adottata e ogni struttura ha i suoi problemi e le sue debolezze)
        \defaultvspace
        \begin{table}[h]
            \centering
            \begin{tabular}{|c|c|c|}
                \hline
                \textbf{Progetto} & \textbf{N\textsuperscript{o} smell Arcan} & \textbf{N\textsuperscript{o} smell Designite} \\
                \hline
                Accumulo & 98 & 71 \\
                Beam & 5 & 5 \\
                Bookkeeper & 52 & 149\\
                Cassandra & 96 & 247\\
                Druid & 6 & 86\\
                Flink & 66 & 112\\
                Geode & 143 & 533\\
                Kafka & 138 & 243\\
                Skywalking & 65 & 109\\
                Zookeeper & 39 & 68\\
                \hline
                Totale & 708 & 1623 \\
                \hline
            \end{tabular}
            \caption{Confronto istanze Subclasses Do Not Redefine Methods}
            \label{tab:caption}
        \end{table}
        \defaultvspace


%Unutilizied Abstraction

\subsubsection{Unutilizied Abstraction}
    Designite è in grado di effettuare la \textit{detection} anche dello \textit{smell} \textit{Unutilizied Abstraction}, rendendo così possibile il confronto con Arcan. I due \textit{tool} presentano strategie di identificazione leggermente differenti, che causano una discrepanza sostanziale nel numero di istanze identificate con 4120 unità in più riscontrate da Designite.

    %Differenza strategia identificazione
    \paragraph{Strategie di identificazione}
        La \textit{detection} di Unutilizied Abstraction da parte del \textit{tool} Designite comporta piccole differenze con le strategie di identificazione individuate per Arcan. L'algoritmo definito da Designite per la \textit{detection} dello \textit{smell} è stato definito come segue:
        \defaultvspace
        \begin{algorithmic}
            \Function{designite-uua-detector}{ }
                \If{il tipo analizzato non ha dipendenze in ingresso}
                     \State{\textbf{return} l'elemento è una Unutilizied Abstraction}
                \Else \If{L'elemento analizzato ha supertipi}
                    \If{Tutti i supertipi non hanno dipendenze in ingresso}
                        \State{\textbf{return} l'elemento è una Unutilizied Abstraction}
                    \EndIf
                \EndIf \EndIf
            \EndFunction
        \end{algorithmic}
        \defaultvspace
        %
        Si può notare che non vengono valutati separatamente i due casi \textit{Unreferenced Abstraction} e \textit{Orphan Abstraction} (presentati nella sezione 4.2) a differenza di Arcan che invece effettua controlli differenti a seconda della tipologia di \textit{abstraction} considerata.
        Non essendoci nessuna distinzione tra tipologia della classe e delle relazioni tra esse, vengono considerati come \textit{smell} tutte le \textit{abstraction} che non presentano dipendenze in ingresso. 
        %Questa differenza può essere data da due motivi in particolare:
        %Sono state formulate due ipotesi differenti per giustificare questa differenza nella detection:
        %\begin{itemize}
        %    \item Non effettuare la distinzione è stata una scelta implementativa degli sviluppatori. È possibile che sia stata dettata anche da un impossibilità ad effettuare controlli separati a causa del parser e della struttura dati utilizzati.
            
        %    \item il parsing effettuato da Designite e la sua architettura permettono di avere solamente dipendenze di un certo tipo per certe classi, perciò un interfaccia senza dipendenze significa che non è stata implementata da nessuno (a differenza di Arcan dove vengono rilevati diversi tipi di nodi, ad esempio dependsOn, anche per le interfacce).
        %\end{itemize}
        %\\
        Viene effettuato inoltre un altro controllo per stabilire se una classe è inutilizzata o meno, attraverso una valutazione dei supertipi della classe sotto analisi. Se la \textit{abstraction} che stiamo considerando presenta dei supertipi, la presenza dello \textit{smell} è verificata, oltre che dalla eventuale assenza di dipendenze in ingresso, anche dalle dipendenze dei supertipi stessi. Se nessuno di essi ne presenta in ingresso allora anche la classe sotto analisi è considerata \textit{unutilizied}. Più in generale si può affermare che Designite considera come tali tutte le \textit{abstraction} figlie di classi senza dipendenze in ingresso. 
        
    % Risultati
    \paragraph{Analisi risultati}
        Nella tabella 12 vengono riportati i dati relativi alla \textit{detection} effettuata con i due diversi \textit{tool}. Da questi si evince che le istanze dello \textit{smell} trovate da Designite sono in numero maggiore rispetto ad Arcan, con una differenza di 4120 unità. Un'analisi approfondita del funzionamento dei due \textit{tool} ha dimostrato che la causa principale di questo divario può essere ricercata nel \textit{parsing} degli elementi e nelle strutture dati utilizzate. La verifica è stata effettuata selezionando le \textit{abstraction} individuate come \textit{smell} da Designite ma non considerate tali da parte di Arcan, e analizzando successivamente la struttura del grafo delle dipendenze al fine di valutare le relazioni della classe dal punto di vista di Arcan.
        È stata riscontrata la presenza di diverse classi nel grafo con archi in ingresso e nessun supertipo, perciò a causa delle relazioni presenti non sono state individuati da Arcan come \textit{unutilizied} e la loro rilevazione da parte di Designite non è causata del controllo sui supertipi. Di conseguenza si può concludere che con molta probabilità sia presente una differenza tra le strutture dati e gli algoritmi di \textit{parsing} utilizzati, con le dipendenze che vengono identificate in maniera diversa causando un grande divario di istanze dello \textit{smell} tra i due \textit{tool}.
        
        Inoltre un altro fattore che potrebbe influenzare il numero delle istanze è la considerazione come \textit{smell} da parte di Designite di tutti i figli di classi inutilizzate indistintamente dal loro utilizzo o meno. Se queste ultime fossero utilizzate, non verrebbero infatti considerate da Arcan come problema, creando una discrepanza nei risultati. 

    
        \defaultvspace
        \begin{table}[h]
            \centering
            \begin{tabular}{|c|c|c|}
                \hline
                \textbf{Progetto} & \textbf{N\textsuperscript{o} smell Arcan} & \textbf{N\textsuperscript{o} smell Designite} \\
                \hline
                Accumulo & 177 & 1955 \\
                Beam & 32 & 22\\
                Bookkeeper & 189 & 906\\
                Cassandra & 137 & 1927\\
                Druid & 128 & 203\\
                Flink & 185 & 431\\
                Geode & 304 & 1557\\
                Kafka & 134 & 1118\\
                Skywalking & 683 & 1182\\
                Zookeeper & 64 & 336\\
                \hline
                Totale & 2033 & 6153\\
                \hline
            \end{tabular}
            \caption{Confronto istanze Unutilizied Abstraction}
            \label{tab:caption}
        \end{table}

%Unnecessary Abstraction

\subsubsection{Unnecessary Abstraction}
    Attraverso il \textit{tool} Designite è possibile anche la \textit{detection} dello \textit{smell} \textit{Unnecessary Abstraction}, sebbene con la limitazione di identificare solamente il caso \textit{constant placeholder}, a differenza di Arcan che individua anche \textit{procedural class} e le \textit{over engineered}. Questa diversità tra le capacità dei due \textit{smell} crea un grande divario anche per quanto riguarda il numero di istanze, con Arcan che ne ha individuate 2051 in più.
    
    Il confronto delle strategie di identificazione sarà effettuato analizzando solamente il caso \textit{constant placeholder} mentre l'analisi dei risultati verrà condotta esaminando anche tutti i casi presenti. Nello specifico, saranno inizialmente considerati tutti i casi di \textit{Unnecessary Abstraction}, per poi proseguire con l'analisi solamente di \textit{constant placeholder}.
    
    
    \paragraph{Strategie di identificazione}
        Per la ricerca delle \textit{abstraction} \textit{constant placeholder} gli sviluppatori di Designite hanno effettuato una ricerca di tutte le classi senza metodi e con un numero di attributi minore o uguale a una determinata soglia, che al momento della stesura di questo elaborato assumeva il valore 5. 
        Segue una rappresentazione in pseudocodice dell'algoritmo di \textit{detection}:
        \defaultvspace
        \begin{algorithmic}
            \Function{designite-una-detector}{ }
                \State{def SOGLIA = 5} 
                \If{L'elemento analizzato ha un numero di attributi $<=$ SOGLIA}
                    \If{L'elemento analizzato non ha metodi}
                        \State{\textbf{return} l'elemento è una Unnecessary Abstraction}
                    \EndIf
                \EndIf
            \EndFunction
        \end{algorithmic}
        \defaultvspace \\
        %Gli algoritmi di ricerca implementati dai due tool presentano alcune differenze.
        %Oltre alla differenza degli algoritmi di ricerca tra le casistiche considerate, di cui è stato già discusso in precedenza, la sola \textit{detection} del caso \textit{constant placeholder} presenta ulteriori diversità. 
        \\
        Mentre Arcan considera come \textit{constant placeholder} le \textit{abstraction} che presentano solamente attributi costanti a prescindere dal loro numero, Designite introduce una soglia minima di attributi che deve definire una classe per essere considerata una \textit{constant placeholder}, poiché gli sviluppatori del tool devono aver ritenuto che classi con pochi costante non possano essere considerati \textit{placeholder}. Il problema delle classi costanti con pochi attributi è stato riscontrato in Arcan, dove è emerso che la causa principale della presenza di falsi positivi nel caso \textit{constant placeholder} fosse proprio la loro presenza nel sistema.
        
        Un ulteriore differenza tra i due algoritmi si può ricercare nel controllo degli attributi poiché, mentre Arcan verifica che gli siano costanti e con un valore già assegnato, Designite si limita solamente a verificare che la \textit{abstraction} considerata non contenga altro oltre ad essi, effettuando così la \textit{detection} di classi che non hanno tutti valori costanti.
         
         Infine la struttura di Arcan ha reso necessario anche il filtraggio delle classi rappresentanti errori, mentre per Designite non è stato necessario poiché probabilmente l'introduzione della soglia esclude queste classi che solitamente contengono massimo due attributi costanti.
    
    \paragraph{Analisi risultati}
        Essendo condotta l'analisi nei due modi differenti, nella tabella riepilogativa delle istanze è stata inserita una nuova colonna, denominata \textit{N\textsuperscript{o} constant placeholder}, che indica il numero di istanze di questa tipologia trovate da Arcan nei vari progetti.
        
        Dai dati presentati nella tabella 13 emergono i diversi approcci adottati dai due tool. La differenza di 2051 istanze in favore di Arcan è in soprattutto conseguenza della ricerca delle \textit{procedural class} che ha un forte impatto sul numero totale di \textit{smell} trovati. \\
        
        \defaultvspace
        \begin{table}[h]
            \centering
            \begin{tabular}{|c|c|c|c|}
                \hline
                \textbf{Progetto} & \textbf{N\textsuperscript{o} smell} & \textbf{N\textsuperscript{o} constant} & \textbf{N\textsuperscript{o} smell} \\
                & \textbf{Arcan} & \textbf{placeholder} & \textbf{designite}\\
                \hline
                Accumulo & 1113 & 24 & 77 \\
                Beam & 13 & 24 & 2\\
                Bookkeeper & 96 & 15 & 74\\
                Cassandra & 639 & 4 & 33\\
                Druid & 52 & 5 & 7\\
                Flink & 47 & 25 & 28\\
                Geode & 166 & 19 & 34\\
                Kafka & 119 & 19 & 51\\
                Skywalking & 208 & 36 & 128\\
                Zookeeper & 38 & 7 & 6\\
                \hline
                Totale & 2491 & 178 & 440 \\
                \hline
            \end{tabular}
            \caption{Confronto istanze Unnecessary Abstraction}
            \label{tab:caption}
        \end{table}
        Se anche per Arcan non venissero considerati i due casi \textit{over engineered} e \textit{procedural class}, si può notare come Designite sia in grado di trovare un numero maggiore di istanze. 
            %Questa differenza può essere riflesso del fatto che la ricerca di Arcan effettua controlli più approfonditi sugli attributi.
            %Questa differenza può essere un riflesso del fatto che, come analizzato nel paragrafo precedente, nella ricerca Designite non effettua il controllo che tutti gli attributi della classe siano valori già assegnati e costanti, e per questo motivo un gran numero di classi escluse da Arcan sono considerate come smell invece da Designite. Anche per questo smell è stata effettuata un'analisi delle classi trovate da Designite, al fine di enfatizzare meglio le differenze. Questa analisi ha confermato quello detto in precedenza, e cioè che non viene effettuato nessun controllo sugli attributi finale. Inoltre ha evidenziato un'altra differenza fondamentale tra i due tool: mentre Arcan controlla solamente gli attributi definiti dalla classe sotto esame, Designite considera anche tutti quelli ereditati da eventuali supertipi. Questo causa quindi la presenza di molte classi vuote che, essendo sottotipi di \textit{unnecessary smell}, sono anch'esse affette dallo smell; in Arcan questo non succederebbe. Inoltre non vengono considerati i valori di default
        %Alternative
        Un'analisi dettagliata delle classi individuate da quest'ultimo ha permesso di evidenziare le possibili cause principali che hanno portato a questa discrepanza:
            %Un'analisi delle classi trovate da Designite ha permesso di evedenziare due delle cause che hanno portato a questa differenza tra i due tool:
        \begin{itemize}
            \item Il \textit{tool} Designite non effettua il controllo che tutti gli attributi della classe siano costanti e con valori di default assegnati e l'analisi effettuata ha confermato questa differenza, in quanto tra le istanze erano presenti numerose classi che presentavano attributi non costanti.
            
            \item C'è una differenza per quanto riguarda gli attributi considerati all'interno delle classi, poiché mentre Arcan effettua l'analisi solamente degli attributi della \textit{abstraction} presa sotto analisi, Designite considera anche tutti quelli ereditati da eventuali supertipi. Questa diversità porta quindi la \textit{detection} da parte di Designite di classi vuote che sono sottotipi di \textit{costant placeholder}, che non verrebbero invece individuate da Arcan.
        \end{itemize}
        
        

%Conclusioni
\subsubsection{Conclusioni}
La sezione 5.4 ha illustrato un confronto tra i due \textit{tool} Arcan e Designite, in grado di effettuare la \textit{detection} degli smell \textit{Subclasses Do Not Redefine Methods}, \textit{Unutilizied Abstraction} e \textit{Unnecessary Abstraction}.

Designite ha dimostrato una capacità maggiore di riconoscimento di istanze per gli smell \textit{SR} e \textit{UUA} ma poi ne ha individuato un numero minore per quanto riguarda lo smell \textit{UNA}. Non essendo presenti i valori di \textit{precision} per Designite, non è stato possibile fare un confronto approfondito riguardo questo aspetto degli algoritmi. In generale, è stata constatata sia la presenza di diversi falsi positivi di Arcan non segnalati da Designite sia istanze individuate da Designite che non sono risultate come smell.

Riguardo \textit{Subclasses Do Not Redefine Methods}, Designite è in grado di discriminare i metodi ereditati in base al modificatore mentre Arcan li considera tutti, anche se ritengo un caso abbastanza isolato la presenza in una gerarchia di due metodi \textit{private} definiti da classi diverse che condividono lo stesso nome. Arcan inoltre non considera nella ricerca diversi falsi positivi relativi a \textit{exception}, riconosciuti invece da Designite. Per questi motivi introdotti, ritengo che nessun \textit{tool} prevalga sull'altro in considerazione alla qualità e capacità della \textit{detection}.
    %Per questi motivi introdotti, ritengo che la \textit{detection} di Arcan per questo smell sia più accurata, anche a fronte di una \textit{precision} di 89.75 \%.

In merito a \textit{Unutilizied Abstraction}, Designite non effettua alcuna distinzione tra la tipologia di classi analizzate e le relazioni che intercorrono tra esse, a differenza di quanto effettuato da Arcan. Questa situazione porta quindi a non considerare il caso delle \textit{orphan abstractions}, introdotto nella definizione dello smell come una delle due tipologie di manifestazione (sezione 4.2). Molte interfacce e classi astratte, che dovrebbero risultare come smell, non vengono erroneamente considerate da Designite.
%Questo causa la detection anche di interfacce e classi astratte non implementate oppure estese, come richiede lo smell, ma soltanto utilizzate attraverso la definizione di attributi di quel tipo oppure l'utilizzo di metodi e costanti definite al loro interno.
Inoltre la ricerca dei figli di classi inutilizzate introduce diverse istanze in più non prese in considerazione da Arcan. Il controllo di Arcan sembra quindi più completo, a fronte di un'analisi maggiormente approfondita su classi e relazioni e una fedeltà superiore alla definizione dello smell rispetto a Designite.
    % La loro mi sembra sbagliata, la ns. ha una precision alta 95.43 \%

Infine, relativamente a \textit{Unnecessary Abstraction} si può affermare che Arcan effettua un controllo più dettagliato e accurato, poiché vengono ricercati nel codice tutti e tre i casi di classi non necessarie presentati, sacrificando però la \textit{precision} (73.66\%) influenzata soprattutto dalle classi procedurali non considerate da Designite. Anche valutando solamente il caso \textit{constant placeholder}, Arcan esegue verifiche più approfondite riguardo le \textit{abstraction} sotto analisi.

Concludendo, si può affermare che nonostante la ricerca di Designite sia in grado di trovare un maggior numero di istanze, Arcan effettua controlli più dettagliati sulle classi e interfacce sotto analisi. Non disponendo però dei dati di \textit{precision} di tutte e due i \textit{tool}, non può essere effettuato un confronto sulla differenza della reale efficacia dei due algoritmi.




% Non sempre più vuol dire meglio, di designite non abbiamo i valori di precision

% Designite ne trova di più
% Designite considera alcuni FP di Arcan ma Arcan considera alcuni casi come FP di cui Designite non effettua la detection
