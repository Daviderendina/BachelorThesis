\section{Lavori correlati}
% descrivi brevementi dei lavori della letteratura sulla detection di AS....fatti dare da Ilaria  uno dei nostri articoli e quello mio con Azadi

%Questo capitolo presenta diversi lavori appartententi alla letteratura degli \textit{smell}, che trattano temi riguardanti la loro \textit{detection}, le differenti strategie di \textit{refactoring} e la loro influenza sulla qualità di un progetto.

Questo capitolo illustra diversi lavori, appartenenti alla letteratura degli \textit{smell}, che riguardano la loro \textit{detection}, le diverse tecniche di \textit{refactoring} e l' influenza sulla qualità del progetto. La prima sezione presenta nel dettaglio i lavori riguardanti il tema degli \textit{smell}, mentre saranno catalogati nella successiva numerosi \textit{tool} in grado di effettuare il riconoscimento di \textit{architecture smell}, con enfasi sulle loro capacità di \textit{detection}, sulle strutture dati utilizzate e strategie di ricerca adottate.

%La letteratura sugli \textit{smell} presenta numerosi lavori riguardanti il riconoscimento di code smell \cite{fowler2018refactoring}, analisi e descrizione di tool per la detection automatica degli stessi, anche se la parte riguardante gli \textit{architectural smell} è tra le più esigue \cite{AzadiFontana}. Questo capitolo presenterà una parte dei lavori facenti parte della letteratura degli AS, più in dettaglio verrano presentati articoli e libri riguardanti il loro riconoscimento e analisi e più in generale sulle architetture software e i problemi che le affliggono. 
%Successivamente seguirà un catalogo di tool per l'analisi e la detection di AS, con una breve analisi delle loro capacità e peculiarità nella detection.

%In questa sezione verranno presentati alcuni lavori della letteratura sulla detection degli AS. Inizialmente, verranno trattati alcuni articoli sulla teoria degli AS, la loro classificazione e più in generale sul tema delle architetture software. Successivamente verranno presentati vari tool per l'analisi e la detection statica di AS e verranno brevemente analizzate le loro peculiarità di detection.\\
\\
 %%% SMELL IN CORSIVO %%%

\subsection{Bad smell e refactoring}
La letteratura degli \textit{smell} contiene diversi lavori riguardanti la loro definizione, l'impatto che hanno sulla qualità del sistema e le tecniche per il loro riconoscimento. La maggioranza di questi lavori però riguarda la categoria dei \textit{code smell} \cite{fowler2018refactoring}, in quanto meno lavoro è stato svolto nell'ambito degli \textit{architectural smell} \cite{AzadiFontana}.

%Introduzione code smell
Martin Fowler et al. \cite{fowler2018refactoring} hanno per primi introdotto i concetti di \textit{code smell} e \textit{refactoring}, cioè problemi comuni all'interno del codice e le relative tecniche di rimozione. Gli autori hanno voluto creare una guida per effettuare il \textit{refactoring} in diverse situazioni, in modo da evitare l'introduzione di \textit{bug} e mantenere un'elevata qualità del codice.

Diversi aspetti riguardanti le tecniche di \textit{refactoring} e le sue differenti applicazioni sono stati anche approfonditi M. Lippert et al. \cite{lippert2006refactoring}. I due autori trattano anche il tema degli \textit{architecture smell}, presentando un loro catalogo e diverse strategie per effettuare il \textit{refactoring} in maniera ottimale. Oltre al tema degli \textit{architecture smell}, vengono discusse anche tecniche applicate in altri ambiti come \textit{API}, \textit{database} e progetti complessi (\textit{Large Refactoring}).

Il tema del \textit{refactoring} è trattato anche da M. Stal \cite{stal2014refactoring}, che affronta il tema della prevenzione dell'erosione architetturale attraverso l'applicazione di differenti tecniche per la rimozione di \textit{smell}.

% definizione di AS
La definizione del concetto di \textit{architecture smell} è stata effettuata da J. Garcia et al. \cite{Garcia2009}, stabilendo gli aspetti che caratterizzano questa tipologia di problemi architetturali e le differenze tra essi e gli anti-pattern architetturali. Il loro lavoro inoltre contiene una descrizione dettagliata di quattro diversi \textit{architectural smell}, contenenti anche esempi generici di diagrammi \textit{UML} al fine di favorire il lavoro dei progettisti e sviluppatori nella ricerca degli stessi.

G. Suryanarayana et al. \cite{SURYANARAYANA201521} hanno proposto un catalogo di \textit{architecture smell} suddivisi in 4 categorie differenti in base al principio di progettazione \cite{booch2008object} violato dallo \textit{smell}: \textit{Abstraction}, \textit{Encapsulation}, \textit{Hierarchy} e \textit{Modularization}. Oltre alla definizione, ogni \textit{smell} presenta inoltre informazioni riguardanti strategie di \textit{refactoring} consigliate, impatto sulla qualità del codice, cause potenziali e diversi esempi.

% Dipendenze tra gli smell
F. Arcelli Fontana et al. \cite{arcelli2015codesmellrelations} hanno studiato le relazioni tra diverse tipologie di \textit{code smell} all'interno di 74 sistemi, al fine di valutare il loro impatto sul \textit{technical debt}. Gli autori hanno riscontrato una percentuale significante di casi di correlazione tra istanze di \textit{smell} differenti e ciò ha permesso di confermare che i \textit{code smell} tendono a presentarsi in gruppo, influenzando negativamente la manutenzione dei progetti e il debito tecnico in maniera superiore rispetto alla loro manifestazione singola.

Una ricerca empirica riguardante l'influenza dei \textit{code smell} sul degrado architetturale di un sistema è stata svolta anche da I. Macia et al. \cite{macia2014codeanomalies}, attraverso lo studio di 40 versioni di 6 sistemi software e l'analisi di 2056 anomalie del codice. Questa ricerca ha evidenziato che il 78\% dei problemi architetturali presenti nei programmi possono essere ricondotti ad anomalie del codice e che le strategie di \textit{refactoring} non sempre contribuiscono in maniera significativa alla rimozione di problemi collegati all'architettura.

Le possibili correlazioni e dipendenze tra \textit{code smell} e \textit{architectural smell} sono stati approfondite da F. Arcelli et al. \cite{arcelli2019dependency}. Attraverso l'analisi delle correlazioni tra 19 \textit{code smell} e 4 \textit{architectural smell}, gli autori sono stati in grado di dimostrare che le due categorie considerate possono considerarsi tra loro indipendenti e che quindi è necessario prestare attenzione al \textit{refactoring} di tutte le categorie di \textit{smell}.

M. Tufano et al. \cite{tufano2015whenwhy} hanno effettuato una ricerca riguardante il momento e le motivazioni dell'introduzione di \textit{code smell} nel progetto, attraverso l'analisi di diverse \textit{commit} effettuate su 200 progetti \textit{open source} provenienti da differenti ecosistemi. La ricerca effettuata ha permesso agli autori la definizione di quattro differenti situazioni comuni riguardanti l'introduzione degli smell, con relativi consigli e rimedi per evitare questi scenari.


Duc Minh Le et al. \cite{mihn2018architecturaldecay} sono stati in grado di analizzare l'impatto degli \textit{architectural smell} sul sistema, attraverso lo studio delle relazioni tra gli \textit{smell} trovati e i problemi indicati dagli \textit{issue trackers} di diversi progetti. Lo studio ha dimostrato il forte impatto che la loro presenza ha sul decadimento del software, poiché rende necessaria una quantità di risorse considerevole per garantire il mantenimento del software durante il suo ciclo di vita. 
Inoltre i \textit{file} nel progetto coinvolti negli \textit{smell} risultano maggiormente inclini agli errori e alle continue modifiche rispetto a quelli che non presentano problemi.



%\subsection{Tool per il riconoscimento di architectural smell}
    % Devo fare un introduzione migliore ai tool per l'analisi architetturale e più in particolari agli AS
\subsection{Tool per il riconoscimento di architectural smell}

Sul mercato sono disponibili diversi \textit{tool} per effettuare analisi statica di sistemi software, in grado di analizzare differenti linguaggi di programmazione, effettuare il calcolo di numerose metriche e ricercare \textit{architecture smell} a diverse granularità.

U. Azadi et al. \cite{AzadiFontana} hanno presentato un catalogo di nove \textit{tool} disponibili e non sul mercato, volti alla \textit{detection} di \textit{architectural smell}. Gli autori hanno posto molta enfasi sul confronto operativo dei vari \textit{tool}, analizzando in particolare le differenze tra le diverse regole di \textit{detection} e i differenti risultati ottenuti. 
Verranno proposti di seguito diversi \textit{tool} per la \textit{detection} di \textit{architectural smell}, mettendo in risalto le loro capacità e la peculiarità del funzionamento e della ricerca di ognuno di essi, la maggior parte dei quali presenti nel lavoro di Azadi \cite{AzadiFontana}.

%Designite
%\paragraph{Designite} 
    Designite \cite{Designite} è un \textit{tool} che permette la valutazione della qualità del software attraverso l'identificazione di un vasto range di \textit{smell} (ovvero \textit{code}, \textit{architectural} oppure \textit{design smell}), il calcolo di diverse metriche e l'identificazione del codice replicato nel progetto. %Questo tool fornisce inoltre la possibilità di monitorare il numero di smell in differenti versioni dello stesso software, per l'analisi dell'andamento del processo di sviluppo e manutenzione del sw stesso.\\
    Designite effettua il \textit{parsing} del codice in modo da generare un \textit{Abstract Syntax Tree (AST)}, utilizzato poi per la creazione del meta-modello necessario al funzionamento del \textit{tool}. L'analisi di questo meta-modello permette poi di effettuare la \textit{detection} e calcolare le metriche.
    %computazione di differenti metriche (utilizzate anche per la detection degli smell) oppure direttamente la detection degli smell, attraverso un'analisi approfondita del meta-modello stesso. 


%AI Reviewer
%\paragraph{AI Reviewer} 
    AI Reviewer \cite{AIReviewer} è un \textit{tool} in grado di analizzare progetti C++ al fine di trovare, anche attraverso la ricerca di \textit{architectural smell}, violazioni dei principi \textit{S.O.L.I.D.} introdotti da Robert Martin \cite{martin2002agile} e calcolare diverse metriche a granularità differenti. Il funzionamento è basato sulla rappresentazione astratta e dettagliata del progetto derivata dal \textit{parsing} del codice sorgente attraverso un modello, analizzato da un ulteriore componente per svolgere i differenti calcoli.

% Massey Architecture Explorer
%\paragraph{Massey Architecture Explorer} 
    Massey Architecture Explorer (MAE) \cite{MasseyArchitectureExplorer} è un applicazione \textit{browser-based} per visualizzare e analizzare architetture di progetti Java. MAE estrae un modello \textit{graph-based} dal \textit{bytecode} Java, analizzato poi da un algoritmo apposito per la \textit{detection} di \textit{architecture smell} e anti-pattern architetturali.
    %che viene poi 'analizzato' da un algoritmo per la ricerca di antipattern architetturali relativi ai problemi di refactoring (?). 


% Arcade
%\paragraph{ARCADE} 
    ARCADE \cite{Arcade7180083} è un software in grado di monitorare e analizzare le modifiche e il decadimento architetturale di un progetto attraverso le sue differenti versioni. 
    ARCADE può recuperare l'architettura dal codice sorgente del progetto, per poi utilizzare le informazioni estratte per il calcolo di metriche (confrontando l'architettura estratta con quella di tutte le versioni precedenti e successive di quel software), per effettuare analisi statistiche sui dati ottenuti e ricercare diversi \textit{architectural smell}. \cite{AzadiFontana}. 


% Stan (Dipendenze Cicliche)
%\paragraph{STAN} 
    STatic ANalizer for Java \cite{STAN} è un software per effettuare analisi strutturali di progetti Java analizzando il \textit{bytecode} del progetto. Con STAN è possibile visualizzare le dipendenze tra classi e package del sistema, calcolare metriche per la qualità del software con anche \textit{rating} e indicazioni sui valori ottenuti ed effettuare la detection dello smell \textit{Cyclic Dependency}.
    

% Sonargraph
%\paragraph{Sonargraph}
    Sonargraph \cite{sonargraph}\cite{SonargraphArticle} è un software che permette il monitoraggio e l'analisi di architettura software e metriche di un progetto, focalizzato sulla riduzione del \textit{technical debt}. Tramite un \textit{Domain Specific Language} (DSL) specializzato, gli sviluppatori possono definire le regole per la descrizione della loro architettura, controllate e validate in maniera automatica durante il processo di sviluppo software. Supporta diversi linguaggi di programmazione (Java, C#, C, C++, Python) ed è possibile inoltre la detection di due smell architetturali \cite{AzadiFontana}.
    
    
% Hotspot Detector
%\paragraph{Hotspot Detector} and detected by the combination of history and architecture in- formation
    Hotspot Detector \cite{HotspotDetector} è un \textit{tool} per la detection automatica di cinque problemi architetturali, derivati dalla teoria di progettazione di Baldwin e Clark \cite{BaldwinClark_DesignRulesVolume1} e ricercati attraverso la combinazione di informazioni storiche e architetturali. Gli stessi sviluppatori del \textit{tool} hanno ribattezzato questi problemi come \textit{Hotspot Patterns}.


% Structure101
%\paragraph{Structure101}
    Structure101 \cite{structure101} è un \textit{tool} che permette la visualizzazione dell'architettura del progetto in maniera modulare, gerarchica e organizzata e la \textit{detection} di due differenti \textit{architecture smell}, con simulazione e applicazione delle tecniche di \textit{refactoring}. È inoltre possibile la definizione regole di dipendenza, stratificazione e visibilità attraverso elementi strutturali.




