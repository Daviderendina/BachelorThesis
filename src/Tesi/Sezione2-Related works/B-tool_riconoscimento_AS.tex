%\subsection{Tool per il riconoscimento di architectural smell}
    % Devo fare un introduzione migliore ai tool per l'analisi architetturale e più in particolari agli AS
\subsection{Tool per il riconoscimento di architectural smell}

Sul mercato sono disponibili diversi \textit{tool} per effettuare analisi statica di sistemi software, in grado di analizzare differenti linguaggi di programmazione, effettuare il calcolo di numerose metriche e ricercare \textit{architecture smell} a diverse granularità.

U. Azadi et al. \cite{AzadiFontana} hanno presentato un catalogo di nove \textit{tool} disponibili e non sul mercato, volti alla \textit{detection} di \textit{architectural smell}. Gli autori hanno posto molta enfasi sul confronto operativo dei vari \textit{tool}, analizzando in particolare le differenze tra le diverse regole di \textit{detection} e i differenti risultati ottenuti. 
Verranno proposti di seguito diversi \textit{tool} per la \textit{detection} di \textit{architectural smell}, mettendo in risalto le loro capacità e la peculiarità del funzionamento e della ricerca di ognuno di essi, la maggior parte dei quali presenti nel lavoro di Azadi \cite{AzadiFontana}.

%Designite
%\paragraph{Designite} 
    Designite \cite{Designite} è un \textit{tool} che permette la valutazione della qualità del software attraverso l'identificazione di un vasto range di \textit{smell} (ovvero \textit{code}, \textit{architectural} oppure \textit{design smell}), il calcolo di diverse metriche e l'identificazione del codice replicato nel progetto. %Questo tool fornisce inoltre la possibilità di monitorare il numero di smell in differenti versioni dello stesso software, per l'analisi dell'andamento del processo di sviluppo e manutenzione del sw stesso.\\
    Designite effettua il \textit{parsing} del codice in modo da generare un \textit{Abstract Syntax Tree (AST)}, utilizzato poi per la creazione del meta-modello necessario al funzionamento del \textit{tool}. L'analisi di questo meta-modello permette poi di effettuare la \textit{detection} e calcolare le metriche.
    %computazione di differenti metriche (utilizzate anche per la detection degli smell) oppure direttamente la detection degli smell, attraverso un'analisi approfondita del meta-modello stesso. 


%AI Reviewer
%\paragraph{AI Reviewer} 
    AI Reviewer \cite{AIReviewer} è un \textit{tool} in grado di analizzare progetti C++ al fine di trovare, anche attraverso la ricerca di \textit{architectural smell}, violazioni dei principi \textit{S.O.L.I.D.} introdotti da Robert Martin \cite{martin2002agile} e calcolare diverse metriche a granularità differenti. Il funzionamento è basato sulla rappresentazione astratta e dettagliata del progetto derivata dal \textit{parsing} del codice sorgente attraverso un modello, analizzato da un ulteriore componente per svolgere i differenti calcoli.

% Massey Architecture Explorer
%\paragraph{Massey Architecture Explorer} 
    Massey Architecture Explorer (MAE) \cite{MasseyArchitectureExplorer} è un applicazione \textit{browser-based} per visualizzare e analizzare architetture di progetti Java. MAE estrae un modello \textit{graph-based} dal \textit{bytecode} Java, analizzato poi da un algoritmo apposito per la \textit{detection} di \textit{architecture smell} e anti-pattern architetturali.
    %che viene poi 'analizzato' da un algoritmo per la ricerca di antipattern architetturali relativi ai problemi di refactoring (?). 


% Arcade
%\paragraph{ARCADE} 
    ARCADE \cite{Arcade7180083} è un software in grado di monitorare e analizzare le modifiche e il decadimento architetturale di un progetto attraverso le sue differenti versioni. 
    ARCADE può recuperare l'architettura dal codice sorgente del progetto, per poi utilizzare le informazioni estratte per il calcolo di metriche (confrontando l'architettura estratta con quella di tutte le versioni precedenti e successive di quel software), per effettuare analisi statistiche sui dati ottenuti e ricercare diversi \textit{architectural smell}. \cite{AzadiFontana}. 


% Stan (Dipendenze Cicliche)
%\paragraph{STAN} 
    STatic ANalizer for Java \cite{STAN} è un software per effettuare analisi strutturali di progetti Java analizzando il \textit{bytecode} del progetto. Con STAN è possibile visualizzare le dipendenze tra classi e package del sistema, calcolare metriche per la qualità del software con anche \textit{rating} e indicazioni sui valori ottenuti ed effettuare la detection dello smell \textit{Cyclic Dependency}.
    

% Sonargraph
%\paragraph{Sonargraph}
    Sonargraph \cite{sonargraph}\cite{SonargraphArticle} è un software che permette il monitoraggio e l'analisi di architettura software e metriche di un progetto, focalizzato sulla riduzione del \textit{technical debt}. Tramite un \textit{Domain Specific Language} (DSL) specializzato, gli sviluppatori possono definire le regole per la descrizione della loro architettura, controllate e validate in maniera automatica durante il processo di sviluppo software. Supporta diversi linguaggi di programmazione (Java, C#, C, C++, Python) ed è possibile inoltre la detection di due smell architetturali \cite{AzadiFontana}.
    
    
% Hotspot Detector
%\paragraph{Hotspot Detector} and detected by the combination of history and architecture in- formation
    Hotspot Detector \cite{HotspotDetector} è un \textit{tool} per la detection automatica di cinque problemi architetturali, derivati dalla teoria di progettazione di Baldwin e Clark \cite{BaldwinClark_DesignRulesVolume1} e ricercati attraverso la combinazione di informazioni storiche e architetturali. Gli stessi sviluppatori del \textit{tool} hanno ribattezzato questi problemi come \textit{Hotspot Patterns}.


% Structure101
%\paragraph{Structure101}
    Structure101 \cite{structure101} è un \textit{tool} che permette la visualizzazione dell'architettura del progetto in maniera modulare, gerarchica e organizzata e la \textit{detection} di due differenti \textit{architecture smell}, con simulazione e applicazione delle tecniche di \textit{refactoring}. È inoltre possibile la definizione regole di dipendenza, stratificazione e visibilità attraverso elementi strutturali.


